% This file is part of the DDD project.
% Copyright 2014 David W. Hogg (NYU).

\documentclass{beamer}

\renewcommand{\emph}[1]{\textbf{#1}}
\newcommand{\project}[1]{\textsl{#1}}
\newcommand{\Kepler}{\project{Kepler}}

\begin{document}

\begin{frame}
  \frametitle{David W. Hogg}
  \begin{itemize}
  \item \textsl{Professor of Physics,}\\ New York University
  \item \textsl{Adjunct Senior Staff Scientist,}\\ Max-Planck-Institut f\"ur Astronomie
  \item \textsl{Executive Director,}\\ Moore--Sloan Data Science Environment at NYU
  \item \textsl{Deputy Director,}\\ NYU Center for Data Science
  \end{itemize}
\end{frame}

\begin{frame}
  \frametitle{my projects}
  \begin{itemize}
  \item data analysis
    \begin{itemize}
    \item \project{emcee} (top-10 astronomy paper in 2013)
    \end{itemize}
  \item weak gravitational lensing
    \begin{itemize}
    \item \project{The Tractor}
    \end{itemize}
  \item Milky Way formation and dynamics
    \begin{itemize}
    \item \project{Gaia} and \project{APOGEE}
    \end{itemize}
  \item \emph{exoplanet discovery and characterization}
    \begin{itemize}
    \item \ldots
    \end{itemize}
  \item calibration and fundamental astronomy
    \begin{itemize}
    \item \project{Sloan Digital Sky Survey}
    \item \project{Euclid}
    \end{itemize}
  \item citizen science with amateur astronomers
    \begin{itemize}
    \item \project{Astrometry.net}
    \item \project{Open Source Sky Survey}
    \end{itemize}
  \end{itemize}
\end{frame}

\begin{frame}
  \frametitle{extra-Solar planets: How we find them}
  \begin{itemize}
  \item direct imaging
  \item radial velocity
  \item \emph{transit}
    \begin{itemize}
    \item Earth analog: \emph{84~ppm} signal for 13~h, once per year
    \end{itemize}
  \item astrometry
  \end{itemize}
\end{frame}

\begin{frame}
  \frametitle{NASA \Kepler\ Satellite}
  \begin{itemize}
  \item stared at one patch of sky for 4~yr, exposure every 30~min
  \item only downlinked tiny fraction of $8\times 10^7$ pixels
  \item $10^{11}$ total measurements on $10^5$ stars plus meta-data
  \item optimal for stable photometric measurements
    \begin{itemize}
    \item 10~ppm precision
    \item \emph{pessimal} for self-calibration
    \end{itemize}
  \item \project{TESS}
  \end{itemize}
\end{frame}

\begin{frame}
  \frametitle{Earth and Jupiter analogs}
\end{frame}

\begin{frame}
  \frametitle{exoplanet populations}
\end{frame}

\begin{frame}
  \frametitle{exoplanet populations}
  [exopop example here]
\end{frame}

\begin{frame}
  \frametitle{flexible models and marginalization}
\end{frame}

\begin{frame}
  \frametitle{flexible models and marginalization}
  [George example here]
\end{frame}

\begin{frame}
  \frametitle{data-driven self-calibration}
  \begin{itemize}
  \item capitalize on \emph{causal structure} of the problem
  \item massive regression
  \item train-and-test framework
  \end{itemize}
\end{frame}

\begin{frame}
  \frametitle{data-driven self-calibration}
  [PLM example here]
\end{frame}

\begin{frame}
  \frametitle{time-domain astronomy}
  \begin{itemize}
  \item top priority for astrophysics in the next decades
    \begin{itemize}
    \item exoplanets, supernovae, asteroseismology, light echos
    \item \project{LSST}, \project{Gaia}, \project{Euclid}, \project{WFIRST}
    \end{itemize}
  \end{itemize}
\end{frame}

\begin{frame}
  \frametitle{statistics and applied mathematics}
  \begin{itemize}
  \item new MCMC methodologies
    \begin{itemize}
    \item with Jonathan Goodman (NYU Mathematics)
    \end{itemize}
  \item computer vision for astronomy
    \begin{itemize}
    \item with Rob Fergus (NYU Computer Science) and Bernhard Sch\"olkpf (MPI-IS)
    \end{itemize}
  \item fast linear algebra ($n\,\log^2n$) for kernel matrices
    \begin{itemize}
    \item with Sivaram Ambikasaran and Mike O'Neil (NYU Mathematics)
    \end{itemize}
  \item hierarchical modeling for dummies
    \begin{itemize}
    \item with Brendon Brewer (Auckland Statistics)
    \end{itemize}
  \end{itemize}
\end{frame}

\begin{frame}
  \frametitle{reclaim of ``engineering''}
  \begin{itemize}
  \item astronomers tend to think of ``engineering'' as ``hardware''
  \item all projects are \emph{integrated hardware--software systems}
    \begin{itemize}
    \item design, build, operations, data analysis, legacy
    \item my lab is a software+theory+methods shop
    \end{itemize}
  \end{itemize}
\end{frame}

\end{document}
