% This file is part of the DDD project.
% Copyright 2014 David W. Hogg (NYU).

\documentclass{beamer}

\renewcommand{\emph}[1]{\textbf{#1}}
\newcommand{\project}[1]{\textsl{#1}}
\newcommand{\Kepler}{\project{Kepler}}

\begin{document}

\begin{frame}
  \frametitle{David W. Hogg}
  \begin{itemize}
  \item \textsl{Professor of Physics,}\\ New York University
  \item \textsl{Adjunct Senior Staff Scientist,}\\ Max-Planck-Institut f\"ur Astronomie
  \item \textsl{Executive Director,}\\ Moore--Sloan Data Science Environment at NYU
  \item \textsl{Deputy Director,}\\ NYU Center for Data Science
  \end{itemize}
\end{frame}

\begin{frame}
  \frametitle{my projects}
  \begin{itemize}
  \item data analysis
    \begin{itemize}
    \item \project{emcee} (top-10 astronomy paper in 2013)
    \end{itemize}
  \item weak gravitational lensing
    \begin{itemize}
    \item \project{The Tractor}
    \end{itemize}
  \item Milky Way formation and dynamics
    \begin{itemize}
    \item \project{Gaia} and \project{APOGEE}
    \end{itemize}
  \item \emph{exoplanet discovery and characterization}
    \begin{itemize}
    \item \ldots
    \end{itemize}
  \item calibration and fundamental astronomy
    \begin{itemize}
    \item \project{Sloan Digital Sky Survey}
    \item \project{Euclid}
    \end{itemize}
  \item citizen science with amateur astronomers
    \begin{itemize}
    \item \project{Astrometry.net}
    \item \project{Open Source Sky Survey}
    \end{itemize}
  \end{itemize}
\end{frame}

\begin{frame}
  \frametitle{extra-Solar planets: How we find them}
  \begin{itemize}
  \item direct imaging
  \item radial velocity
  \item \emph{transit}
    \begin{itemize}
    \item Earth analog: \emph{84~ppm} signal for 13~h, once per year
    \end{itemize}
  \item astrometry
  \end{itemize}
\end{frame}

\begin{frame}
  \frametitle{NASA \Kepler\ Satellite}
  \begin{itemize}
  \item stared at one patch of sky for 4~yr, exposure every 30~min
  \item only downlinked tiny fraction of $8\times 10^7$ pixels
  \item $10^{11}$ total measurements on $10^5$ stars plus meta-data
  \item optimal for stable photometric measurements (10~ppm precision)
  \item \emph{pessimal} for self-calibration
  \end{itemize}
\end{frame}

\begin{frame}
  \frametitle{data-driven calibration}
  \begin{itemize}
  \item capitalize on \emph{causal structure} of the problem
  \item massive regression
  \item train-and-test framework
  \end{itemize}
\end{frame}

\begin{frame}
  \frametitle{flexible models and marginalization}
\end{frame}

\begin{frame}
  \frametitle{exoplanet populations}
\end{frame}

\begin{frame}
  \frametitle{Earth analogs}
\end{frame}

\begin{frame}
  \frametitle{time-domain astronomy}
\end{frame}

\begin{frame}
  \frametitle{statistics and applied mathematics}
\end{frame}

\begin{frame}
  \frametitle{reclaim of ``engineering''}
\end{frame}

\end{document}
