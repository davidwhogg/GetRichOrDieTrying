% This file is part of the GetRichOrDieTrying project.
% Copyright 2019 David W. Hogg and Megan Bedell

% # Style notes
% - Third person, not first person!!
% - Use macros for acronyms, project names, and so on!
% - Use \nasasection{} not \section{} so we can tweak typography

% # To-do list
% - Write zeroth draft!
% - Put in references.

\documentclass[12pt, letterpaper]{article}
\usepackage{fancyhdr, graphicx}
\setlength{\headsep}{2ex}
%----- exact 1-in margins
% NB: headheight and headsep MUST exist and be set
\setlength{\textwidth}{6.5in}
\setlength{\textheight}{9in}
\addtolength{\textheight}{-1.0\headheight}
\addtolength{\textheight}{-1.0\headsep}
\setlength{\topmargin}{0.0in}
\setlength{\oddsidemargin}{0.0in}
\setlength{\evensidemargin}{0.0in}

%----- typeset certain kinds of words
\newcommand{\observatory}[1]{\textsl{#1}}
\newcommand{\package}[1]{\textsf{#1}}
\newcommand{\project}[1]{\textsl{#1}}
\newcommand{\an}{\package{Astrometry.net}}
\newcommand{\NASA}{\observatory{NASA}}
\newcommand{\Kepler}{\observatory{Kepler}}
\newcommand{\kepler}{\Kepler}
\newcommand{\ketu}{\observatory{K2}}
\newcommand{\kt}{\ketu}
\newcommand{\KT}{\ketu}
\newcommand{\ktwo}{\ketu}
\newcommand{\MAST}{\observatory{MAST}}
\newcommand{\EA}{\observatory{Exoplanet Archive}}
\newcommand{\TESS}{\observatory{TESS}}
\newcommand{\tess}{\TESS}
\newcommand{\galex}{\observatory{GALEX}}
\newcommand{\Spitzer}{\observatory{Spitzer}}
\newcommand{\spitzer}{\Spitzer}
\newcommand{\gaia}{\observatory{Gaia}}
\newcommand{\Gaia}{\gaia}
\newcommand{\lsst}{\observatory{LSST}}
\newcommand{\sdss}{\observatory{SDSS}}
\newcommand{\foreign}[1]{\textit{#1}}
\newcommand{\eg}{\foreign{e.g.}}
\newcommand{\etal}{\foreign{et~al.}}
\newcommand{\etc}{\foreign{etc.}}
\newcommand{\ie}{\foreign{i.e.}}
\newcommand{\vs}{\foreign{vs.}}
\newcommand{\opcit}{\foreign{op~cit.}}

%----- typeset journals
% \newcommand{\aj}{Astron.\,J.}
% \newcommand{\apj}{Astrophys.\,J.}
% \newcommand{\apjl}{Astrophys.\,J.\,Lett.}
% \newcommand{\apjs}{Astrophys.\,J.\,Supp.\,Ser.}
% \newcommand{\mnras}{Mon.\,Not.\,Roy.\,Ast.\,Soc.}
% \newcommand{\aap}{Astron.\,\&~Astrophys.}

%----- Tighten up paragraphs and lists
\setlength{\parskip}{0.0ex}
\setlength{\parindent}{0.2in}
\renewenvironment{itemize}{\begin{list}{$\bullet$}{%
  \setlength{\topsep}{0.0ex}%
  \setlength{\parsep}{0.0ex}%
  \setlength{\partopsep}{0.0ex}%
  \setlength{\itemsep}{0.0ex}%
  \setlength{\leftmargin}{1.5\parindent}}}{\end{list}}
\newcounter{actr}
\renewenvironment{enumerate}{\begin{list}{\arabic{actr}.}{%
  \usecounter{actr}%
  \setlength{\topsep}{0.0ex}%
  \setlength{\parsep}{0.0ex}%
  \setlength{\partopsep}{0.0ex}%
  \setlength{\itemsep}{0.0ex}%
  \setlength{\leftmargin}{1.5\parindent}}}{\end{list}}

%----- mess with paragraph spacing!
\makeatletter
\renewcommand\paragraph{\@startsection{paragraph}{4}{\z@}%
                                    {1ex}%
                                    {-1em}%
                                    {\normalfont\normalsize\bfseries}}
\makeatother

%----- Special Hogg list for references
  \newcommand{\hogglist}{%
    \rightmargin=0in
    \leftmargin=0.25in
    \topsep=0ex
    \partopsep=0pt
    \itemsep=0ex
    \parsep=0pt
    \itemindent=-1.0\leftmargin
    \listparindent=\leftmargin
    \settowidth{\labelsep}{~}
    \usecounter{enumi}
  }

%----- side-to-side figure macro
%------- make numbers add up to 94%
 \newlength{\figurewidth}
 \newlength{\captionwidth}
 \newcommand{\ssfigure}[3]{%
   \setlength{\figurewidth}{#2\textwidth}
   \setlength{\captionwidth}{\textwidth}
   \addtolength{\captionwidth}{-\figurewidth}
   \addtolength{\captionwidth}{-0.02\figurewidth}
   \begin{figure}[htb]%
   \begin{tabular}{cc}%
     \begin{minipage}[c]{\figurewidth}%
       \resizebox{\figurewidth}{!}{\includegraphics{#1}}%
     \end{minipage} &%
     \begin{minipage}[c]{\captionwidth}%
       \textsf{\caption[]{\footnotesize {#3}}}%
     \end{minipage}%
   \end{tabular}%
   \end{figure}}

%----- top-bottom figure macro
 \newlength{\figureheight}
 \setlength{\figureheight}{0.75\textheight}
 \newcommand{\tbfigure}[2]{%
   \begin{figure}[htp]%
   \resizebox{\textwidth}{!}{\includegraphics{#1}}%
   \textsf{\caption[]{\footnotesize {#2}}}%
   \end{figure}}

%----- deal with pdf page-size stupidity
\special{papersize=8.5in,11in}
\setlength{\pdfpageheight}{\paperheight}
\setlength{\pdfpagewidth}{\paperwidth}

% no more bad lines!
\sloppy\sloppypar

% A better underline!
% tex.stackexchange.com/questions/36894/underline-omitting-the-descenders
\usepackage[T1]{fontenc}
\usepackage[latin1]{inputenc}
\usepackage{soul}
\usepackage{xcolor}
\usepackage{xparse}
\makeatletter
\ExplSyntaxOn
\cs_new:Npn \white_text:n #1
  {
    \fp_set:Nn \l_tmpa_fp {.01}
    \fp_mul:Nn \l_tmpa_fp {#1}
    \llap{\textcolor{white}{\the\SOUL@syllable}\hspace{\fp_to_decimal:N \l_tmpa_fp em}}
    \llap{\textcolor{white}{\the\SOUL@syllable}\hspace{-\fp_to_decimal:N \l_tmpa_fp em}}
  }
\NewDocumentCommand{\whiten}{ m }
    {
      \int_step_function:nnnN {1}{1}{#1} \white_text:n
    }
\ExplSyntaxOff

\NewDocumentCommand{ \varul }{ D<>{5} O{0.2ex} O{0.1ex} +m } {%
\begingroup
\setul{#2}{#3}%
\def\SOUL@uleverysyllable{%
   \setbox0=\hbox{\the\SOUL@syllable}%
   \ifdim\dp0>\z@
      \SOUL@ulunderline{\phantom{\the\SOUL@syllable}}%
      \whiten{#1}%
      \llap{%
        \the\SOUL@syllable
        \SOUL@setkern\SOUL@charkern
      }%
   \else
       \SOUL@ulunderline{%
         \the\SOUL@syllable
         \SOUL@setkern\SOUL@charkern
       }%
   \fi}%
    \ul{#4}%
\endgroup
}
\makeatother


% Underline paragraph titles
\usepackage[explicit]{titlesec}
\titleformat{\paragraph}[runin]
    {\normalfont\normalsize\bfseries}{\theparagraph}{1em}
    {\varul{#1}}

  \renewcommand{\headrulewidth}{0pt}
  \pagestyle{fancy}
  \lhead{\textsf{Hogg \& Bedell / Extreme-precision radial-velocity in the presence of stellar variability}}
  \rhead{\textsf{\thepage}}
  \cfoot{}
\begin{document}\sloppy\sloppypar\raggedbottom\frenchspacing

\nasasection{Introduction}

\noindent
From a fundamental perspective, there is essentially no limit to how
precisely astronomers might measure the radial velocity of the center of mass
of a distant star with an extreme precision radial-velocity (\EPRV) spectrograph.
Of course in real life, there are real limits, set
by (for example) photon noise, stellar surface convection and activity,
variations in the atmosphere, and spectrograph flexure and illumination. In practice,
essentially no stars have had their center-of-mass velocities measured
consistently with an empirical scatter (\acronym{RMS} around the mean,
say) smaller than $1\,\mps$.

At the same time, \EPRV\ is responsible for hundreds of \foreign{ab initio}
exoplanet discoveries and many hundreds more \foreign{a posteriori}
characterizations of planets
discovered elsewhere. That is, \EPRV\ is hugely successful.
Because most of the discoveries and characterizations have been made
at or near precision limits---that is, near what is detectable at noise levels
greater than $1\,\mps$ per epoch---\textbf{any improvement in precision directly
translates into increased scientific return on investment},
both from existing archival data and from new data from new programs.

(Extensive words about how data-driven models are changing stellar science, and
that we two are at the forefront of that.)

\paragraph{Relevance to \NASA\ \XRP\ objectives:} The 2018 \acronym{NSPIRES}
Amendment 68 appendix E.5 (Second Exoplanets Research) description of this
\XRP\ call lists four general categories for proposals (page
E.5-1). This proposal falls into two of these categories. It is
designed to improve the capabilities of all spectroscopic facilities
performing radial-velocity measurments relevant to exoplanet observations.
Thus, it is designed to ``Detect exoplanets and/or confirm exoplanet candidates'',
and it is designed to ``Observationally characterize exoplanets, their
atmospheres, or specific host star properties that directly impact our
understanding of the exoplanetary system''. That is, the proposed activities
are precisely aligned with \XRP\ goals.

This proposal is about making ground-based spectroscopic programs more
precise, more capable, and more productive for exoplanet research
goals. The \XRP\ call for proposals explicitly includes support for
ground-based observing programs, including those at public, private,
and \NASA-supported facilites.

This project is directly aimed at improving the capabilities of the
\NEID\ Spectrograph, which sees first light this year. This
spectrograph is part of the \NASA\ \NNEXPLORE\ program. Thus this
proposal is written to develop capabilities that will directly support
and improve the legacy value of existing \NASA\ missions. It will also
make possible new observing programs and strategies for community
proposers to the \NNEXPLORE\ facility.

Finally, in the context of the US Decadal Survey and proposals for near-future
\NASA\ missions, the question arises: Should we be doing \EPRV\ from space?
The answer to this question depends critically on the detailed error budget on
\EPRV\ measurements. If \EPRV\ measurements have uncertainties dominated by
instrument stability, atmosphere, and tellurics, then space is the place to be.
\textbf{If \EPRV\ measurements are dominated by stellar astrophysical noise sources---the
objects of study in this proposal---then taking \EPRV\ to space is not a good
use of \NASA\ resources}. Thus the mitigation of astrophysical noise sources is
of critical importance to \NASA's near-term and middle-term planning and strategy
for its exoplanet missions and programs.
Page E.5-2 of the \acronym{NSPIRES} Amendment 68 says ``Proposals should
demonstrate relevance to NASA by describing the benefit for NASA
missions, with specific past, current, or future missions or programs
identified.'' Consider that demonstrated, here and below.

\paragraph{Relevance to \NASA\ Strategic Plans:}
\begin{itemize}
\item
\textit{\NASA\ Strategic Plan 2018}:
The entire Exoplanet Exploration program (of which \XRP\ is a part) falls under
Strategic Objective 1.1 (Understand the Sun, Earth, Solar System, and Universe)
of Strategic Goal 1 (Expand Human Knowledge Through Scientific Discoveries) of the
\NASA\ Strategic Plan. Because this project is directly
addressing the operation of the ground-based \NNEXPLORE\ spectrograph operated
through novel partnerships,
it also connects (perhaps weakly) to Strategic Objective 4.1 (Engage in Partnership Strategies).
\item
\textit{\NASA\ Strategic Space Technology Investment Plan}:
This project has some limited relevance to \NASA's plans for technology development.
Results related to the advisability of performing \EPRV\ work in space are related to
the prioritization of improving technology readiness for spectrograph components.
\item
\textit{Voyages: Charting the Course for Sustainable Human Space Exploration}:
Since the strategic plan for \NASA\ human exploration does not yet include plans
for planets outside the Solar System, we admit that our relevance to this part
of the \NASA\ Strategy is weak!
\end{itemize}

\nasasection{First Stage: Information about \EPRV\ in time-varying stars (information)}

Important to bring up at the end of this section that when the star
varies, the CCF no longer contains all of the RV information in the
spectrum!

\nasasection{Second Stage: Mitigation of asteroseismic variability (p-modes)}

\nasasection{Third Stage: Mitigation of myriad other variabilities (stochastics)}

\nasasection{Timeline and project management}

Although the three parts of this project are presented as ``stages'',
they will not be staged serially.
Rather we will run all three in parallel to make best use of our
personnel.
All three stages will involve PI Hogg, Co-PI Bedell, and the project
graduate student student (GRA). 

The First Stage (information) will be led by PI Hogg, with experiments (the
toy-model experiments) performed by the GRA, and writing by the full
team.
It will occupy parts of Years~1 and 2, producing a paper
in each of those two years.
One paper will focus on the information-theoretic approach to the problem,
and the second will focus on the causal-inference approaches.

The Second Stage (p-modes) will be led by Co-PI Bedell, with
experiments performed by the GRA, and writing by the full team.
The scientific analysis will take place in Years~1 and 2, with
writing and dissemination in Years~2 and 3.
This stage will also produce two papers, one focusing on passive
approaches (nulling), and one focusing on active approaches (resolving
and fitting).
The dissemination part will also include two community workshops on
\EPRV\ as a time-domain problem, in which we build on our experience
of working with radial-velocity teams to bring new methods and
software practices to the community.
These workshops will happen in Years~2 and 3.
They are important to the project because some of the approaches we
will be advocating will be ``outside the box'' for some projects.

The Third Stage (stochastics) will have different components led by PI
Hogg, Co-PI Bedell, and the GRA, depending on particular interests of
the GRA.
Each of the sources of stochastic variability will have different
signatures in time--spectrum space (as discussed above); we will use
Year~1 to do exploratory work in the model residuals to set priorities
(which projects will be easier to see in the data, which harder).
That prioritization will create a schedule for projects, with data
analyses and experiments performed by the GRA, and projects supervised
by PI Hogg and Co-PI Bedell as it makes sense.
Some of the exploratory experiments will produce publishable results
to write up in Year~2 and more holistic mitigation will require
methods and software that will be built in Years~2 and 3 and published
and disseminated in Year~3.
Again, we will use the community workshops in Years~2 and 3 as part of
the dissemination

PI Hogg will be responsible for overall project management. This is a
tiny project; most of the management will involve decisions about our
priorities given the experiments we perform in Year~1 and the
challenges we discover there.
PI Hogg will also be responsible for academic supervision of, and
career mentoring for, the GRA.
The two workshops will be chaired by Co-PI Bedell, with PI Hogg and
the GRA on the organizing committees.

\nasasection{Prior \NASA\ support}

\nasasection{Frequently asked questions}

\paragraph{Why this team?}

\paragraph{Related: How can these two influence the whole community?}

\paragraph{Why do this now?}

\paragraph{Hasn't this already been done?}

\paragraph{Isn't it all just about getting better instrument stability and calibration?}

\paragraph{You can't ever really get rid of stochastic stellar noise, right?}

\paragraph{How does this connect to \XRP\ program priorities?}

Three things: (1)~detection and characterization of exoplanets in
general, across many projects and communities. (2)~suppport of the
existing \NNEXPLORE\ program goals. (3)~answering the hard question
about taking \EPRV\ to space.

\end{document}
