\documentclass[12pt]{article}
\begin{document}

There are three \emph{Major Activities} proposed as part of the
planetary systems direction of this proposal: Discovery,
Characterization, and Population Inference.

\paragraph{Planet Discovery:}
This Major Activity revolves around the
extreme observational precision required to detect planets around
their host stars. The technical challenge emerges from the fact that
stars are far more massive and far more luminous than the
planetary systems they host.

In the \emph{imaging domain}, planets can appear because (with luck)
they can induce \emph{periodic transit signals} in an (unresolved) star's light
curve (brightness as a function of time). For Earth-like planets
around Sun-like stars, these transits block (only) $10^{-4}$ of the
star's light, and only for about a dozen hours once every hundreds of
days. Detection of these signals requires long time baselines, extreme
photometric precision, and suppression or modeling of both spacecraft and stellar
variability, which (in most cases) are larger in amplitude than the
transits. In imaging, exoplanets can also appear as \emph{microlensing
  events}, which are brighter events but which are much rarer, and
essentially never repeat. Here the challenge is not so much
suppression of noise as confident and rapid identification of
non-repeating microlensing against a background of stellar flaring and
other variability (and data artifacts). With the launch of the
\textsl{Gaia} Spacecraft, it will be possible to see planets in the
imaging domain as \emph{astrometric wobble}.

In the \emph{spectroscopic domain}, planets appear as
\emph{radial-velocity perturbations} on the rest-frame velocity of
their host star. Here the detection challenge is that stellar surfaces
are seething with convection and quasi-coherent oscillation modes,
leading to stochastic and time-structured radial-velocity variations
that are larger (once again) than the most interesting planetary
signals. There are also currently unexplained issues with spectrograph
radial-velocity calibration that may be contributing to---or even
dominating---radial-velocity noise.

Our approaches to these problems, in keeping with the physical themes
of this proposal, are to divide the incoming effects into different
causal categories (CITE BS), and model each cause with as much
flexibility and physical input as possible. Spacecraft or instrument
variations or problems imprint themselves coherently onto many
observations of different, independent stars, and can therefore be
separated (in principle) from astrophysical signals. We have been
pioneers in making extremely flexible models of spacecraft variability
(CITE DUN, DFM, DFMK2, ANGUS), but we have only scratched the surface,
because our spacecraft models have been \emph{purely} data-driven and
have not made use of the fact that dominant sources of noise come from
instrument pointing and temperature. That is, our high dimensional
linear models will be replaceable with lower-dimensional but nonlinear
model that represent physically possible responses to perturbations.
These models---which are easiest to implement in imaging data---also
need to be taken to spectroscopic data, where there are additional
issues from slit or fiber losses and effects, wavelength calibration
variations, and superimposed calibration signals.

When working at scale, or preparing for population inferences (see
below), it is imperative that planet search and vetting proceed
according to automated methods and involve no human interaction. This
may seem obvious, but few planet detection systems are \emph{truly}
hands-free. We are the community leaders in developing such methods,
where we usually take a generative modeling approach to false-positive
signals like system jumps or drop-outs, cosmic rays, stellar surface
events, and contamination by binary star systems (CITE DFM
LONG-PERIOD). These issues only become more important (and more
important to automate) as searches get more sensitive. Our analysis
suggests that radial-velocity searches could benefit signficantly from
false-positive and system-event models.

We have observed---but not yet capitalized on the fact---that stellar
surfaces can be modeled to reasonable approximation as a
time-correlated linear response to stochastic driving, in a kind of
Green's function or linear ode framework (CITE FENGJI THESIS). If we
make the further assumption that the stochastic driving is Gaussian on
some timescales, there is a physically justifiable Gaussian Process
model for stellar surface variability that can be used in both
imaging and spectroscopic planetary systems discovery projects. These
models are also valuable for characterization (below).

Modeling of exoplanet signals themselves appears trivial, but optimal
sensitivity to small planets might require that models be built of the
aperiodic perturbations to transit and radial-velocity signals induced
by gravitational interactions and orbit evolution. This requires
high-performance dynamical modeling (more below).

In all of the above, when we search we actually fit the instrument,
anomaly, star, and planet models simultaneously, because we have shown
that the flexibility of the models permits over-fitting if each effect
is independently fit and subtracted or suppressed (CITE DFM, DUN
WANG).

\paragraph{Planet characterization:}
This Major Activity involves generating full likelihood
information---or posterior information under (technically vague)
interim priors---about planetary system properties (phase-space state
or system architecture), given all available data. It turns out that
it is just as important to precisely characterize the star as it is to
characterize the planet, because planet properties are measured
relative to star properties.

On star characterization, we have been pioneering data-driven methods
to precisely label stars with relevant physical parameters,
capitalizing on large spectroscopic data sets and small numbers of
stars with good physical labels (CITE CANNON; CITE CASEY). These
methods are data-driven, but physically informed, because they are
incredibly flexible, but constrained to produce dependencies on
physical parameters that are physically reasonable. They also
incorporate the heteroskedastic noise models provided by the
spectroscopic experiments. These methods have only scratched the
surface, because they need to be generalized to take data from
heterogeneous mixtures of spectrographs and photometers, and account
for differing spectral resolutions and wavelength domains. They also
need to accept prior information from atomic physics and new data
sources such as the \textsl{Gaia} Mission data which will come
available in the next few years.

On planetary system characterization, one challenge is that the most
interesting systems involve gravitational interactions that must be
modeled \emph{inside the MCMC} iteration (CITE THINGS). That is, they
require very fast n-body codes; our experience is that these can and
should be high-order integrators (CITE?), but sometimes these
integrators need to be modified to output data on fine time-steps;
after all, the whole point of a high-order integrator is that you can
take long steps. This is a rich area for new research. It is also the
case that the same kind of dynamics that goes into the integrators
also goes into the MCMC code: The leading inference systems use
Hamiltonian MCMC (CITE THINGS), along with analytically differentiated
code. The relationships between the Hamiltonian structure of the
dynamics and the Hamiltonian structure of the sampling might are worth
exploring further.

Stellar surface variations and spacecraft or telescope issues affect
characterization as well as discovery; the models described above for
discovery matter here too. One direction of interest is mapping
stellar and planetary surfaces with time-domain imaging and
spectroscopy, which has led to a number of breakthroughs (CITE
SUNSPOTS, CITE hot-Jupiter TEMPERATURE MAPS, CITE
ROSSITER-McLAUGHLIN). Again, there is a lot to do here; a transiting
exoplanet makes (effectively) a sparse map of the stellar surface;
understanding this can deliver information about both the star and the
planet.

\paragraph{Population inference}
There are thousands of planetary systems known, but there are still
great controversies---and only imprecise conclusions---about the
abundance of planets and planetary systems of different types (CITE
AND CITE). This Major Activity seeks to address this.

There are manifold reasons that planetary system populations are only
known imprecisely, but some leading causes include the following:
$\bullet$~There aren't yet good and agreed-upon parameterizations for
arbitrary n-body planetary systems; the current description of
multiple-planet systems (around possibly multiple-star hosts) are
physically vague and contain massive degeneracies (including but not
limited to labeling degeneracies). $\bullet$~The catalogs of
exoplanets on which population inferences are created as hard catalogs
with completeness and false-alarm rates estimated; neither of these
rates are confidently measured, and there is no way to perform the
population inference \emph{in the space of the raw
  data}. $\bullet$~Planets are measured with substantial mass and
radius uncertainties, such that a simple completeness-corrected
histogram of distributions is not a good population estimator. Proper
inference requires a hierarchical (multi-level) model (CITE
DFM). $\bullet$~Some of the most interesting planetary systems, and
the ones most like our own Solar System, include dominant planets at
long periods, where no parameters (not even periods) of the planets
are well constrained. The need for sophisticated inference is even
more severe here, and will become only more important with a future of
time-constrained \textsl{TESS} Satellite data coming soon.

We intend to work along all of these directions, in particular
exploring system representations, moving inferences closer to the raw
data where possible, and making hierarchical inferences more
sophisticated and better in performance. We would also like to
constrain population inferences to have stability and consistency
properties, such that the inferred population could really exist in a
dynamically generated Universe. This involves understanding long-term
stability (CITE DECK AND OTHERS), but also conditioning either the
representation or else the density as expressed with the
representation to be invariant with time, or really slowly varying
with time.

Along these lines, it is interesting to find planetary systems that
appear to be unstable or short-lived. These might give important clues
to system formation.

At the present day, planetary system formation models are
computational simulations; the do not put probabilities over
outcomes---they do not generate a likelihood function. We will explore
statistically principled methods (such as ABC; CITE STUFF) for
comparing planet-formation simulations to observed planetary systems;
in principle this could either follow or obviate direct population
inferences. The long-term goal of extra-solar planetary science is to
understand the formation and evolution of our and other planetary
systems.

\end{document}
