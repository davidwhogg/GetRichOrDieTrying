% This file is part of the GetRichOrDieTrying / AdversaryLSS project
% Copyright 2018 the authors.

% to-do
% -----
% - get down to one page

\documentclass[12pt, fullpage, letterpaper]{article}
\usepackage{fancyheadings, graphicx}
\setlength{\headsep}{0ex}
%----- exact 1-in margins
% NB: headheight and headsep MUST exist and be set
\setlength{\textwidth}{6.5in}
\setlength{\textheight}{9in}
\addtolength{\textheight}{-1.0\headheight}
\addtolength{\textheight}{-1.0\headsep}
\setlength{\topmargin}{0.0in}
\setlength{\oddsidemargin}{0.0in}
\setlength{\evensidemargin}{0.0in}

%----- typeset certain kinds of words
\newcommand{\observatory}[1]{\textsl{#1}}
\newcommand{\package}[1]{\textsf{#1}}
\newcommand{\project}[1]{\textsl{#1}}
\newcommand{\an}{\package{Astrometry.net}}
\newcommand{\boss}{\project{BOSS}}
\newcommand{\des}{\project{DES}}
\newcommand{\euclid}{\observatory{Euclid}}
\newcommand{\EA}{\observatory{Exoplanet Archive}}
\newcommand{\gaia}{\observatory{Gaia}}
\newcommand{\Gaia}{\gaia}
\newcommand{\galex}{\observatory{GALEX}}
\newcommand{\Kepler}{\observatory{Kepler}}
\newcommand{\kepler}{\Kepler}
\newcommand{\lsst}{\observatory{LSST}}
\newcommand{\MAST}{\observatory{MAST}}
\newcommand{\sdss}{\observatory{SDSS}}
\newcommand{\Spitzer}{\observatory{Spitzer}}
\newcommand{\spitzer}{\Spitzer}
\newcommand{\TESS}{\observatory{TESS}}
\newcommand{\thetractor}{\package{The~Tractor}}
\newcommand{\wfirst}{\observatory{WFIRST}}
\newcommand{\latin}[1]{\textit{#1}}
\newcommand{\eg}{\latin{e.g.}}
\newcommand{\etal}{\latin{et~al.}}
\newcommand{\etc}{\latin{etc.}}
\newcommand{\ie}{\latin{i.e.}}
\newcommand{\vs}{\latin{vs.}}

%----- math shih
\newcommand{\given}{\,|\,}

%----- typeset journals
% \newcommand{\aj}{Astron.\,J.}
% \newcommand{\apj}{Astrophys.\,J.}
% \newcommand{\apjl}{Astrophys.\,J.\,Lett.}
% \newcommand{\apjs}{Astrophys.\,J.\,Supp.\,Ser.}
% \newcommand{\mnras}{Mon.\,Not.\,Roy.\,Ast.\,Soc.}
% \newcommand{\aap}{Astron.\,\&~Astrophys.}

%----- Tighten up paragraphs and lists
\setlength{\parskip}{0.0ex}
\setlength{\parindent}{0.2in}
\renewenvironment{itemize}{\begin{list}{$\bullet$}{%
  \setlength{\topsep}{0.0ex}%
  \setlength{\parsep}{0.0ex}%
  \setlength{\partopsep}{0.0ex}%
  \setlength{\itemsep}{0.0ex}%
  \setlength{\leftmargin}{1.0\parindent}}}{\end{list}}
\newcounter{actr}
\renewenvironment{enumerate}{\begin{list}{\scriptsize{\textbf{\textsf{(\arabic{actr})}}}}{%
  \usecounter{actr}%
  \setlength{\topsep}{0.0ex}%
  \setlength{\parsep}{0.0ex}%
  \setlength{\partopsep}{0.0ex}%
  \setlength{\itemsep}{0.0ex}%
  \setlength{\leftmargin}{1.0\parindent}}}{\end{list}}

%----- mess with paragraph spacing!
\makeatletter
\renewcommand\paragraph{\@startsection{paragraph}{4}{\z@}%
                                    {1ex}%
                                    {-1em}%
                                    {\normalfont\normalsize\bfseries}}
\makeatother

%----- Special Hogg list for references
  \newcommand{\hogglist}{%
    \rightmargin=0in
    \leftmargin=0.25in
    \topsep=0ex
    \partopsep=0pt
    \itemsep=0ex
    \parsep=0pt
    \itemindent=-1.0\leftmargin
    \listparindent=\leftmargin
    \settowidth{\labelsep}{~}
    \usecounter{enumi}
  }

%----- side-to-side figure macro
%------- make numbers add up to 94%
 \newlength{\figurewidth}
 \newlength{\captionwidth}
 \newcommand{\ssfigure}[3]{%
   \setlength{\figurewidth}{#2\textwidth}
   \setlength{\captionwidth}{\textwidth}
   \addtolength{\captionwidth}{-\figurewidth}
   \addtolength{\captionwidth}{-0.02\figurewidth}
   \begin{figure}[htb]%
   \begin{tabular}{cc}%
     \begin{minipage}[c]{\figurewidth}%
       \resizebox{\figurewidth}{!}{\includegraphics{#1}}%
     \end{minipage} &%
     \begin{minipage}[c]{\captionwidth}%
       \textsf{\caption[]{\footnotesize {#3}}}%
     \end{minipage}%
   \end{tabular}%
   \end{figure}}

%----- top-bottom figure macro
 \newlength{\figureheight}
 \setlength{\figureheight}{0.75\textheight}
 \newcommand{\tbfigure}[2]{%
   \begin{figure}[htp]%
   \resizebox{\textwidth}{!}{\includegraphics{#1}}%
   \textsf{\caption[]{\footnotesize {#2}}}%
   \end{figure}}

%----- deal with pdf page-size stupidity
\special{papersize=8.5in,11in}
\setlength{\pdfpageheight}{\paperheight}
\setlength{\pdfpagewidth}{\paperwidth}

% no more bad lines!
\sloppy\sloppypar

\renewcommand{\headrulewidth}{0pt}
\pagestyle{empty}
\newcommand{\LCDM}{\acronym{LCDM}}

\begin{document}

\paragraph{Project summary:}
Cosmology is a very mature field with a very successful standard model
(\LCDM).
So new projects either have to obtain far more precision in
measurements, or else look for subtle departures of data from
predictions.
At the same time, new observational projects are mapping far larger
volumes and far more distant targets, with consequently noisier and
more complex individual-object data.
As we simultaneously push to more challenging data and greater
precisions, we will encounter systematic errors and complexities that
have never mattered previously.

This project will build general-purpose methods and tools that will
make the next generation of cosmological (galaxy, quasar, and
intensity-mapping) surveys more precise, more powerful, and more
capable of making truly new discoveries:

\textsl{(1)}~The project will build adversarial simulated data that is designed to
defeat current methodologies for finding and subtracting or correcting
for systematic effects of calibration and target selection in
large-scale structure surveys.
And then we will build far more general methods, based on
non-parametric modeling, for survey operation that can defeat the
adversaries.

\textsl{(2)}~The project will build, optimize, and operate a new estimator for
galaxy clustering that obviates binning of objects (galaxies or
quasars or pixels) or pairs into bins; rather it estimates continuous
functional forms for clustering as a function of scale, galaxy mass,
color, and so on.
These continuous functional forms make a better representation of the
clustering than any binning; therefore this estimator can do more with
fewer clustering-model components.
This in turn lets it make more powerful measurements with less data,
and with less of the computation that is used to do uncertainty
propagation.
We will use the estimator to make, for the first time,
bound-saturating measurements of cosmological parameters.

\textsl{(3)}~The project will build methodologies that perform (and perform
ourselves) statistically principled searches of large-scale structure
surveys (current and future) for theory-motivated departures from
predictions of the LCDM model.
The cosmological model is very successful.
But the discovery of a departure at intermediate-to-large scales
(where the physical model is extremely accurate) would be so important
and significant that it is worth investing effort here.
The principled aspects of the project will include pre-registration of
hypotheses (that is, an enumeration of the specific forms and kinds of
deviations) prior to search.
And this tool category is related to the previous two, because we need
to distinguish small departures from systematic effects, and we need
to extract signatures at bound-saturating precision.

\paragraph{Intellectual Merit:}
This project will make NSF-funded cosmology projects (including
\SDSSIV, \DESI, and \LSST) more productive, more precise, and more
capable of making ground-breaking discoveries.
It will deliver both general methods and specific, released,
open-source code that works efficiently on current data and can be
incorporated into current and future analysis pipelines.
It will also establish methods for pre-registration of hypotheses that
will be of general value in cosmology and other areas in the natural
sciences.

In addition to being relevant to \NSF\ Division \acronym{AST}, this project also
connects to Divisions \acronym{OAC} and \acronym{IIS}, and touches on the \NSF\ theme
Harnessing the Data Revolution.

\paragraph{Broader Impacts:}
In executing the project, the investigators will build curricular
materials for undergraduate-to-PhD bridge programs, preparing students
in the physical sciences to have better data-science and statistics
skills and therefore better PhD preparations.
We will test and refine these curricular materials in workshops for
students of broad backgrounds in New York City.

\end{document}
