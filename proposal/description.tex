% This file is part of the K2 Cash Money project.
% Copyright 2015 the authors.

% ## style notes:
% - Use \textbf{} for emphasis of key deliverables or proposed items.

% ## issues:

\documentclass[12pt,preprint]{aastex}
\setlength{\headheight}{2ex}
\setlength{\headsep}{3ex}
%----- exact 1-in margins
% NB: headheight and headsep MUST exist and be set
\setlength{\textwidth}{6.5in}
\setlength{\textheight}{9in}
\addtolength{\textheight}{-1.0\headheight}
\addtolength{\textheight}{-1.0\headsep}
\setlength{\topmargin}{0.0in}
\setlength{\oddsidemargin}{0.0in}
\setlength{\evensidemargin}{0.0in}

%----- typeset certain kinds of words
\newcommand{\observatory}[1]{\textsl{#1}}
\newcommand{\package}[1]{\textsf{#1}}
\newcommand{\project}[1]{\textsl{#1}}
\newcommand{\an}{\package{Astrometry.net}}
\newcommand{\NASA}{\observatory{NASA}}
\newcommand{\Kepler}{\observatory{Kepler}}
\newcommand{\kepler}{\Kepler}
\newcommand{\ketu}{\observatory{K2}}
\newcommand{\kt}{\ketu}
\newcommand{\KT}{\ketu}
\newcommand{\MAST}{\observatory{MAST}}
\newcommand{\EA}{\observatory{Exoplanet Archive}}
\newcommand{\TESS}{\observatory{TESS}}
\newcommand{\tess}{\TESS}
\newcommand{\galex}{\observatory{GALEX}}
\newcommand{\Spitzer}{\observatory{Spitzer}}
\newcommand{\gaia}{\observatory{Gaia}}
\newcommand{\Gaia}{\gaia}
\newcommand{\lsst}{\observatory{LSST}}
\newcommand{\sdss}{\observatory{SDSS}}
\newcommand{\latin}[1]{\textit{#1}}
\newcommand{\eg}{\latin{e.g.}}
\newcommand{\etal}{\latin{et~al.}}
\newcommand{\etc}{\latin{etc.}}
\newcommand{\ie}{\latin{i.e.}}
\newcommand{\vs}{\latin{vs.}}

%----- typeset journals
% \newcommand{\aj}{Astron.\,J.}
% \newcommand{\apj}{Astrophys.\,J.}
% \newcommand{\apjl}{Astrophys.\,J.\,Lett.}
% \newcommand{\apjs}{Astrophys.\,J.\,Supp.\,Ser.}
% \newcommand{\mnras}{Mon.\,Not.\,Roy.\,Ast.\,Soc.}
% \newcommand{\aap}{Astron.\,\&~Astrophys.}

%----- Tighten up paragraphs and lists
\setlength{\parskip}{0.0ex}
\setlength{\parindent}{0.2in}
\renewenvironment{itemize}{\begin{list}{$\bullet$}{%
  \setlength{\topsep}{0.0ex}%
  \setlength{\parsep}{0.0ex}%
  \setlength{\partopsep}{0.0ex}%
  \setlength{\itemsep}{0.0ex}%
  \setlength{\leftmargin}{1.5\parindent}}}{\end{list}}
\newcounter{actr}
\renewenvironment{enumerate}{\begin{list}{\arabic{actr}.}{%
  \usecounter{actr}%
  \setlength{\topsep}{0.0ex}%
  \setlength{\parsep}{0.0ex}%
  \setlength{\partopsep}{0.0ex}%
  \setlength{\itemsep}{0.0ex}%
  \setlength{\leftmargin}{1.5\parindent}}}{\end{list}}

%----- mess with paragraph spacing!
\makeatletter
\renewcommand\paragraph{\@startsection{paragraph}{4}{\z@}%
                                    {1ex}%
                                    {-1em}%
                                    {\normalfont\normalsize\bfseries}}
\makeatother

%----- Special Hogg list for references
  \newcommand{\hogglist}{%
    \rightmargin=0in
    \leftmargin=0.25in
    \topsep=0ex
    \partopsep=0pt
    \itemsep=0ex
    \parsep=0pt
    \itemindent=-1.0\leftmargin
    \listparindent=\leftmargin
    \settowidth{\labelsep}{~}
    \usecounter{enumi}
  }

%----- side-to-side figure macro
%------- make numbers add up to 94%
 \newlength{\figurewidth}
 \newlength{\captionwidth}
 \newcommand{\ssfigure}[3]{%
   \setlength{\figurewidth}{#2\textwidth}
   \setlength{\captionwidth}{\textwidth}
   \addtolength{\captionwidth}{-\figurewidth}
   \addtolength{\captionwidth}{-0.02\figurewidth}
   \begin{figure}[htb]%
   \begin{tabular}{cc}%
     \begin{minipage}[c]{\figurewidth}%
       \resizebox{\figurewidth}{!}{\includegraphics{#1}}%
     \end{minipage} &%
     \begin{minipage}[c]{\captionwidth}%
       \textsf{\caption[]{\footnotesize {#3}}}%
     \end{minipage}%
   \end{tabular}%
   \end{figure}}

%----- top-bottom figure macro
 \newlength{\figureheight}
 \setlength{\figureheight}{0.75\textheight}
 \newcommand{\tbfigure}[2]{%
   \begin{figure}[htp]%
   \resizebox{\textwidth}{!}{\includegraphics{#1}}%
   \textsf{\caption[]{\footnotesize {#2}}}%
   \end{figure}}

%----- deal with pdf page-size stupidity
\special{papersize=8.5in,11in}
\setlength{\pdfpageheight}{\paperheight}
\setlength{\pdfpagewidth}{\paperwidth}

% no more bad lines!
\sloppy\sloppypar

% A better underline!
% tex.stackexchange.com/questions/36894/underline-omitting-the-descenders
\usepackage[T1]{fontenc}
\usepackage[latin1]{inputenc}
\usepackage{soul}
\usepackage{xcolor}
\usepackage{xparse}
\makeatletter
\ExplSyntaxOn
\cs_new:Npn \white_text:n #1
  {
    \fp_set:Nn \l_tmpa_fp {.01}
    \fp_mul:Nn \l_tmpa_fp {#1}
    \llap{\textcolor{white}{\the\SOUL@syllable}\hspace{\fp_to_decimal:N \l_tmpa_fp em}}
    \llap{\textcolor{white}{\the\SOUL@syllable}\hspace{-\fp_to_decimal:N \l_tmpa_fp em}}
  }
\NewDocumentCommand{\whiten}{ m }
    {
      \int_step_function:nnnN {1}{1}{#1} \white_text:n
    }
\ExplSyntaxOff

\NewDocumentCommand{ \varul }{ D<>{5} O{0.2ex} O{0.1ex} +m } {%
\begingroup
\setul{#2}{#3}%
\def\SOUL@uleverysyllable{%
   \setbox0=\hbox{\the\SOUL@syllable}%
   \ifdim\dp0>\z@
      \SOUL@ulunderline{\phantom{\the\SOUL@syllable}}%
      \whiten{#1}%
      \llap{%
        \the\SOUL@syllable
        \SOUL@setkern\SOUL@charkern
      }%
   \else
       \SOUL@ulunderline{%
         \the\SOUL@syllable
         \SOUL@setkern\SOUL@charkern
       }%
   \fi}%
    \ul{#4}%
\endgroup
}
\makeatother


% Underline paragraph titles
\usepackage[explicit]{titlesec}
\titleformat{\paragraph}[runin]
    {\normalfont\normalsize\bfseries}{\theparagraph}{1em}
    {\varul{#1}}


% specials for this proposal
\newcommand{\shortauthor}{Hogg, Lang \& Foreman-Mackey}
\newcommand{\fulltitle}{Ultra-precise photometry in crowded fields: A self-calibration approach.}
\renewcommand{\shorttitle}{Ultra-precise photometry:~A self-calibration approach}

% the usuals
\pagestyle{myheadings}
\markright{\textsf{\shortauthor~/~\shorttitle}}
\usepackage{url}
\usepackage{hyperref}

\begin{document}

The \ktwo\ Campaign 9 (K2C9) data will present new challenges, hereby
unmet by the \ktwo\ team and the \kepler\ community.
The critical difference between K2C9 and previous Campaigns (and the
\kepler\ Main Mission) is that the stars of importance will not be
\emph{isolated}.
That is, the success of K2C9 depends on the success of photometry in a
crowded field.
This means that aperture photometry (the basis of almost all
\kepler\ science to date) will no longer be an option.
It also means (as we will describe below) that we need (for the first
time) to understand the detector flat-field at high precision.

There are two general approaches to crowded-field photometry (in our view):
The first is, essentially, to perform \emph{full forward modeling} of the
focal plane, in which the data are fit with a model consisting of a
large number of point sources, a (pixel-convolved) point-spread
function, a DC (or tilted) sky background, and so on.
In this approach, the light curves of the stars will be model parameters
in a baroque model of the whole sky and focal plane.
The second is, essentially, to perform \emph{image differencing}.
In this approach, changes to any star's brightness is seen as a
residual difference between an individual exposure and a reference
image, made by accounting for changes to the telescope pointing (and
PSF, if necessary).

\paragraph{Precise photometry in crowded fields requires an instrument model:}

In this proposal, we are neutral between these approaches for extracting
science from crowded-field imaging.
Our point is that \emph{no matter what kind of crowded-field photometry
we use, we need a high-precision estimate of the detector flat-field}.

In the case of full forward modeling this is perhaps obvious:
The detailed data model is a convolution of the scene with the
pixel-convolved PSF (hearafter ``pcPSF''\footnote{From here on, all references to the
  ``pcPSF'' should be read as ``pixel-convolved PSF'' or what the
  \kepler\ team sometimes calls the ``pixel response function'' or
  PRF.}), multiplied by the flat-field map.
At the precision of \kepler\ data---often parts in $10^5$---the
flat-field matters immensely:
Typical detectors have flat-field variations in the percent range.

In the case of image differencing, we are slightly less sensitive to the
pcPSF but still very sensitive to the flat-field:
When we difference two images, we have to account for small positional
shifts (expected to be fraction-of-a-pixel in the case of K2C9).
If the flat-field is unknown or wrong, the shift of the stars relative
to the pixels does not manifest itself as a pure shift, because light
is moving from one pixel to another, and those pixels have different
sensitivities.
Therefore, to perform image differencing at the information-theory
limit set by the photon noise (the Cram\'er-Rao bound), we need to
understand the flat-field.
And our failure or residuals in the flat-field will translate directly
into noise in the difference image.
That noise will be highest where the image is most crowded---that's where, for
the microlensing applications that are the target of this proposal,
the signals are most likely to appear.

In principle, to perform photmetry at the information-theory limit,
we need to know not just the flat-field but also the pcPSF, the bias,
the dark (or zero), and properties of the sky.
(If you really want to go to town, there are (in principle) two different
flat-fields, one that applies to sky photons and one that applies to star photons,
because of the differences in how the sky and a star illuminate the
s/c.)
Here we propose to infer everything we need to perform precise
photometry, but with a strong emphasis on the stellar flat-field, which we
think will lead to dominant systematic-noise terms in the extracted
photometry, given current techniques and spacecraft (s/c) knowledge.
For example, as long as sources are bright, the star flux dominates
over the background; in addition, if images are being differenced, the
difference signal itself is not expected to be a crowded field, even though
the direct image is.
These properties of the K2C9 microlensing data, thought of as an
image-differencing project, make it much more sensitive to the flat-field than
any other aspect of the s/c imaging.

One important and relevant thing---but which is \emph{not} part of this
proposal---is that everything we achieve in K2C9 will improve photometric
measurements in other Campaigns, at least where they overlap the
telemetry region of K2C9.
Also, all of the techniques and codes we develop will be released
open-source (as is all the work we do), so they can be employed to
infer s/c properties for other Campaigns and other missions.

\paragraph{We perform forward modeling of the s/c to learn instrument model parameters:}

DWH: pcPSF-convolved scene, sky background, sky and star flats.  Striping.  Etc.

\paragraph{The flat-field probably has intra-pixel variations that are non-negligible:}

What one might call the ``first-order'' assumption about a detector is
that it has one scalar sensitivity value per pixel, but otherwise the
pixels are identical.
(The ``zeroth-order'' assumption would be that every pixel has the same
sensitivity!)
In reality, we know that pixels have non-trivial sensitivity maps as a
function of sub-pixel location, so each pixel is \emph{not} identical,
not just in sensitivity, but in shape or detailed sensitivity map.
If we think about expanding the pixel sensitivity, order by order, the
``second-order'' assumption about a detector is that there is some dipole
distortion of each pixel away from the fiducial or mean pixel shape.
A dipole is a vector quantity, so the second-order flat-field has
three numbers per pixel: A scalar ``monopole'' and a vector
``dipole''.
(The dipole is like a mean pixel offset away from the pure rectangular
grid.)
The ``third-order'' flat-field would also have a quadrupole tensor
distortion, and so on.

That is, there is no such thing as a pure ``pixel-convolved PSF'' but
rather there will be a slightly different pcPSF,
appropriate for each pixel.
However, in the limit of small intra-pixel variations in sensitivity,
it is possible to represent the pcPSF for pixel $i$ as a
linear combination of the fiducial pcPSF and two
derivatives with respect to position; these three components are
coadded in proportion to the monopole and two dipole coefficients
that are appropriate for pixel $i$.
This is a principled way of building up, order-by-order, more and more
sophisticated flat-field information, and (relatedly) pcPSF
information.
Similarly, this operation can be applied to a pcPSF-convolved scene of
arbitrary origin (that is, it doesn't require a scene model that is
a mixture of delta-functions).
At second order, the effective scene illuminating the device is a
linear combination of three scenes, with a different set of three
coefficients for each pixel, which are the monopole and dipole
coefficients.
(At third order, it would be a linear combination of six scenes, and
so on.)

We propose in this project to learn or infer the flat-field to second
order, to assess whether there is evidence in the data to support such
an increase in flat-field model complexity, and to see if that upgrade
leads to an increase in the performance of our models of the data.
We have showed that inference at this level is possible in one of our
\kepler\ two-wheel-era white papers (Hogg \etal, 2013, arXiv:1309.0653).
In particular, we showed that images with smearing or dithering reveal
information about the sub-pixel flat-field (in that work we used a
$2\times 2$ subsampled flat-field model, not a multipole expansion,
but the principles are the same).

\paragraph{We might want adjustments to s/c attitude management:}

The principal requirement for the success of any self-calibration project
is that there be proper \emph{redunancy} across observations.
In the relevant self-calibration context, ``redundancy'' is achieved
by having the same stars observed in different pixels, and having the
same pixels subject to very different illumination.
This requires dithering, and preferably dithering that is random (or
pseudo-random in various ways; Holmes, Hogg, \& Rix, 2012, PASP, 124,
1219)
For this reason, it is important to the success of this project that
the s/c \emph{not} point perfectly!

Indeed, our simulations suggest that it would be good for the stellar
scene to ``travel'' over a (roughly) $2\times2$ pixel patch over the
course of (or during some substantial part of) K2C9.
As the K2 team has improved its understanding of the s/c and its
attitude management in the two-wheel era, it has become very good at
locking down the pointing to within a pixel, with the solar-angle
management and thruster activity every $\sim 6$\,h.
However, we might be better off in understanding the flat-field with
slightly \emph{worse} attitude management, because we want the scene
to travel relative to the detector by \emph{more than} one pixel to
see the full range of pixel sensitivity.
(We would want even more travel if the device were better sampled!)

The attitude variability we want could be achieved by any number of
changes to current procedures.
In one class of solutions, we could just reduce the frequency of the
thruster events for the entire duration of K2C9.
In another, we could perform a set of dithering maneuvers in the first
few days of K2C9, to provide a baseline dither set for calibration
purposes.
In another, we could deliberately adjust the s/c so that it is not as
close as possible to the zero-torque attitude.
Obviously many of the considerations involved here are far above the 
``pay grade'' of these poposers.
However, we would be enthusiastic about having an opportunity to influence
s/c attitude management, and the pay-off for calibration might be immense.

It is worth emphasizing here that \emph{if there are no changes to s/c
  attitude management, it will not be possible to self-calibrate the
  K2C9 data} to the levels required for optimal photometric monitoring.
At very low cost, small adjustments to attitude policies could greatly
improve the precision of K2C9's microlensing science.

\paragraph{Project management and timeline:}

As per instructions, this proposal is not being delivered with a complete
budget.
However, our estimated budget is 98,590\,USD.
This budget includes one year of graduate-student support,
10,000\,USD of travel budget to get the team together at NASA Ames a
few times for hacking and sprinting with the K2 team,
and 2,000\,USD of publication support to cover page charges for
the inevitable scientific papers in the refereed literature.

There are two important stages to this project:
The first, which happens well before the start of K2C9 observations,
is to build a small-scale K2C9 simulation, based on what we know from
all the K2 data to date, and use this as a sandbox for writing the
self-calibration code.
In this first stage, we would intend to get working methods and code,
and a solid, written description of both.
The second stage, which begins right at the beginning of K2C9 data
telemetry, is to run this code on the data as soon as it is
telemetered, and perform the relevant visualizing, debugging,
patching-and-rerunning, and so on, to deliver to the community the
flat-field and any other determined self-calibration parameters.
(In particular, we expect that a good model of the ``stripy sky''
might be of great value.)

It is CampHogg standard practice that we don't develop tools without
also being \emph{users} of those tools.
For this reason, in both stages, we would prepare a version of our
image-differencing package (\project{Duh}; https://github.com/davidwhogg/Duh)
to be ready for use on the K2 data.
We would modify \project{Duh} to work with the second-order flat-field
information (this is a very straightforward adjustment), and test
\project{Duh} on the data as it comes in.
Success with \project{Duh} would be scientfically valuable in its own
right, but for the purposes of this proposal, it will be necessary as
an end-to-end functional test of our self-calibration outputs.
That is, the point of this project is \emph{not} to deliver the image
differencing software, but we anticipate that such development will be
necessary to build confidence in our calibration results.

\paragraph{We bring unique skills and expertise to the K2C9 team:}

Our team is uniquely situated to help with the success of K2C9:\\
$\bullet$ We are the first (and only) group to algorithmically
generate a full exoplanet catalog with \ktwo\ data, along with
completeness information (Foreman-Mackey \etal, 2015,
arXiv:1502.04715).\\
$\bullet$ Our group is the leading group in (optical) astronomical
image modeling, which we have used to perform Bayesian weak lensing
inference (Schneider \etal, 2014, arXiv:1411.2608) and
massive forced photometry with the \project{WISE} and \project{SDSS}
missions (Lang, Hogg, \& Schlegel, 2014,
arXiv:1410.7397).\\
$\bullet$ We have been developing a literature on the technical
principles underlying self-calibration, in terms of project design
(Holmes \etal, \opcit) and statistical inference
(Sch\"olkopf \etal, 2015, arXiv:1505.03036).\\
$\bullet$ We put in two extensive white papers in the call following
the second reaction-wheel failure (Hogg \etal, \opcit; and
Montet \etal, 2013, arXiv:1309.0654), and the PI has
been involved in reviewing and advising the \kepler\ Mission.\\
$\bullet$ We have released open-source code for MCMC methods
(\project{emcee}), astrometric calibration (\project{Astrometry.net}),
astronomical image modeling (\project{The Tractor}), \ktwo\ photometry
and exoplanet discovery (\project{ketu}), and Gaussian Processes
(\project{george}).  We have an extensive and clear track record of
delivering valuable open-source code to the community.

We care!

\end{document}
