\documentclass[11pt, letterpaper]{article}
\usepackage{fancyheadings, color, hyperref}

% hypertex insanity
  \definecolor{linkcolor}{rgb}{0,0,0.4}
  \hypersetup{
    colorlinks=true,        % false: boxed links; true: colored links
    linkcolor=linkcolor,    % color of internal links
    citecolor=linkcolor,    % color of links to bibliography
    filecolor=linkcolor,    % color of file links
    urlcolor=linkcolor      % color of external links
  }

% other colors
  \definecolor{grey}{rgb}{0.5,0.5,0.5}
  \newcommand{\deemph}[1]{\textcolor{grey}{\footnotesize{#1}}}

% Margins and spaces
  \setlength{\oddsidemargin}{0in}
  \setlength{\topmargin}{0in}
  \setlength{\headsep}{0.20in}
  \setlength{\headheight}{0.25in}
  \setlength{\textheight}{9.00in}
%  \addtolength{\textheight}{-\headsep}
%  \addtolength{\textheight}{-\headheight}
  \addtolength{\topmargin}{-\headsep}
  \addtolength{\topmargin}{-\headheight}
  \setlength{\textwidth}{6.50in}
  \setlength{\parskip}{0.5ex}
% Headings and footing
  \renewcommand{\headrulewidth}{0pt}
  \pagestyle{fancy}
  \lhead{\deemph{David W. Hogg}}
  \chead{\deemph{Extracting Everything from Astronomical Imaging}}
  \rhead{\deemph{\thepage}}
  \cfoot{}
\newcommand{\arxiv}[1]{\href{http://arxiv.org/abs/#1}{arXiv:#1}}

\begin{document}\sloppy\sloppypar

\noindent\textbf{1. What will be the five-year impact of my work?}
% on one or more of the natural sciences? Is there a key, fundamental
% question that you are trying to answer? How will you measure progress
% towards answering this question over the five years?
\smallskip

If I tried to boil down my projects to a single question, it would be
this:
Can we make our discoveries and measurements in astronomical imaging
at the fundamental precision dictated by the photon arrival rates?
Of course this question is uninteresting without specific contexts.
Two contexts that drive me are in the search for extra-Solar planets
(exoplanets) and in the measurement of cosmological parameters.
In both domains, the signals we care most about are tiny (in any
dimensionless sense) and the astrophysics community is \emph{not}
reaching the photon limit; not even close!

In the exoplanet domain, the context that most strongly drives data
analysis is the search for \emph{Earth analogs}.
In the search for Earth analogs, the NASA \textsl{Kepler} mission has
found many astounding Earth-sized rocky planets.
The \textsl{Kepler} mission finds these by noticing the light that a
(properly aligned) exoplanet blocks when it passes in front of its
host star (from our point of view).
Some of the \textsl{Kepler}-discovered planets are conceivably habitable
(\textit{eg,} Quintana \textit{et al}, \arxiv{1404.5667})
and some of them orbit Sun-like stars
(\textit{eg,} Petigura \textit{et al}, \arxiv{1311.6806}).
None of them are \emph{both}: We have yet to see a rocky planet in the
habitable zone of a Sun-like star.
This isn't because the data aren't good enough:
There are easily enough photons to do the job.
The problems are that the instrument isn't well enough calibrated, the
stars are stochastically variable, and the data searches are
computationally expensive.

We have identified at least four different lines of research that each
could improve the precision of searches or measurements in the
\textsl{Kepler} data, and I intend to pursue all four.

The first is pixel-level (or even sub-pixel) data-driven calibration.

The second line of research is on optimized photometric estimators.

The third line of research is on Gaussian-Process models for stochastic stellar variability.

The fourth line of research is on building physical models for telescope--camera combinations.

In the cosmological domain, the context that drives my work is weak
lensing.

Here the idea is that 


Key questions: Can we find a true Earth analog?  Can we demonstrate
some non-trivial areas where we saturate the photon-limited bounds on
inference in imaging projects?  Can we dramatically improve the
calibration of telescopes and cameras?  Can we develop and propagate
these ideas and standards to the community as a whole?

Making \textsl{Kepler} more sensitive.

(Obviating calibration programs for big projects.)

Taking cosmological inference down to the raw pixel level.  Weak
lensing as an example of this.

Saturating known bounds on inference; in astrophysics you can often
compute (roughly) these bounds.  Conversations with JPL about this
general point.

Evaluation: Are we improving uncertainties on known small exoplanets
around Sun-like stars?  Are we increasing the periods and decreasing
the radii of discovered exoplanets?  Are we doing better and better on
GREAT3 and GREAT4 and so on?  Are projects adopting self-calibration
programs and duplicating or generalizing our methods?

\bigskip
\noindent\textbf{2. How will I advance data-science methodologies?}
% such as statistics, machine learning, automated inference, etc., to
% achieve this goal? We are particularly interested in the ways that
% the data science methodologies that you propose to develop can be
% applied to other fields beyond the one you focus on and
% shared. Please discuss these plans. What work products do you plan
% to make open source?}
\smallskip

Developing and propagating scalable Gaussian Processes.

Building radical self-calibrations with racks and racks of regressions.

Bringing ideas of causal inference into astrophysics.

Working out ideas about probabilistic inference in enormous parameter spaces,
for weak lensing and image modeling generally.

Absolutely everything I do is out in the open; all code is released
with open-source licensing (principally MIT, sometimes GPLv2).
My research day is blogged daily at
\url{http://hoggresearch.blogspot.com/}.
I also have a blog for unpublished ideas that I would like to see
implemented.
All my grant proposals and papers in preparation and code bases in
every state of development are up on the web in open code
repositories.
Even this proposal is visible publicly at
\url{http://github.com/davidwhogg/DDD}.

\end{document}

Please submit a three-page supplement to address the two questions
below, keeping the PDF document to standard margins and 11pt type as
in the pre-application.
 
1. What do you envision as the five-year impact of your work on one or
more of the natural sciences? Is there a key, fundamental question
that you are trying to answer? How will you measure progress towards
answering this question over the five years?
 
2. How will you advance data science methodologies, such as
statistics, machine learning, automated inference, etc., to achieve
this goal? We are particularly interested in the ways that the data
science methodologies that you propose to develop can be applied to
other fields beyond the one you focus on and shared. Please discuss
these plans. What work products do you plan to make open source?

Your application materials will be used by staff and an external
review panel to select about thirty finalists; notifications will be
sent out by July 1st.  Those chosen as finalists will be invited to
give twenty minute talks at the Gordon and Betty Moore Foundation in
Palo Alto, California, on July 28th and 29th. Reasonable travel
expenses will be reimbursed. We anticipate selecting the approximately
15 awardees by late August.

Your FluidReview account at http://ddd.fluidreview.com has been
reactivated to upload your materials. You may also upload an updated
bio-sketch, but this is not required. Please send any questions to
DDDsemifinalists@moore.org.
 
We look forward to receiving your full application by May 12th at noon
Pacific Time.
