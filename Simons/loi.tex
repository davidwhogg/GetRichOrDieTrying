\documentclass[12pt]{article}
\usepackage{fancyheadings, color, hyperref}

% hypertex insanity
  \definecolor{linkcolor}{rgb}{0,0,0.4}
  \hypersetup{
    colorlinks=true,        % false: boxed links; true: colored links
    linkcolor=linkcolor,    % color of internal links
    citecolor=linkcolor,    % color of links to bibliography
    filecolor=linkcolor,    % color of file links
    urlcolor=linkcolor      % color of external links
  }
% Margins and spaces
  \setlength{\oddsidemargin}{0in}
  \setlength{\topmargin}{0in}
  \setlength{\headsep}{0.20in}
  \setlength{\headheight}{0.25in}
  \setlength{\textheight}{9.00in}
  \addtolength{\topmargin}{-\headsep}
  \addtolength{\topmargin}{-\headheight}
  \setlength{\textwidth}{6.50in}
  \setlength{\parskip}{0.5ex}
% Headings and footing
  \renewcommand{\headrulewidth}{0pt}
  \pagestyle{fancy}
  \lhead{\deemph{David W. Hogg}}
  \chead{\deemph{Extracting Everything from Astronomical Imaging}}
  \rhead{\deemph{\thepage}}
  \cfoot{}

\begin{document}

\section*{Simons center for astrophysics and}

\textbf{}

\section*{Introduction}

Modern science increasingly relies on finding subtle but fundamental signals in masses 
of noisy data.
As one example, the Higgs particle discovery rests on a mass of statistical modeling 
and computation.
Several actively debated issues in astrophysics rest on statistical analysis of noisy data.
One is the BISON (... David? ..).
Another is the possible abundance of earth analogue planets, with different estimates 
differing by large factors.


\section*{Our contributions}
Hogg had build several productive collaborations with computational mathematicians, including
Goodman and O'Neil.
In both cases, the work created computational methods and quality software that have enabled
new science.
The Emcee Hammer package that made possible (....David ..?). 
It is a sampler for multivariate probability distributions such as those that arise
as posteriors in Bayesian statistical analyses.
It has the property of {\em affine invariance}, which means its performance is uneffected  
by linear changes of variable.
This allows sampling from distributions that are arbitrarily ill-conditioned.
Such distributions arise whenever components of a random object have different
units (time, distance, mass, angle), and also when the data impose approximate relations 
between variables without bounding variables individually.

The Emcee Hammer package (and ???) have grown into a large well-tested software system
that has dozens of active contributors and hundreds of users.
Several developments, including parallel MCMC, were driven by users who were not
initially collaborators (Dan, is this true?).

Bayesian posterior sampling is sometimes impractical because the forward model is too 
expensive to allow good sampling even with a very good MCMC algorithm.
One example is Gaussian process models with many data points that require a determinant
of a large dense matrix ($10^4\times 10^4$?) each likelihood call.
The fast linear package ??(name)?? made this possible for the first time.
That fast determinant software was subsequently used by (?? other team) for XXX.

\section*{New research}
\subsection*{New samplers **too many words here**}
We have several ideas for much better MCMC samplers.
One, which is partly developed, is an MCMC version of the Gauss Newton method for
nonlinear least squares.
It is common to fit parameters by optimizing least squares fits of nonlinear models to data.
Sophisticated Gauss Newton Marquardt algorithms use user developed software to evaluate the 
model and the Jacobian matrix of model sensitivities.
We have an MCMC algorithm that uses this derivative information to become affine invariant.
The proposals are Gaussian centered on the Gauss Newton point.
We have discovered a robust backoff strategy related to line search without which the 
algorithm is extremely bad, but with which far out-performs the Emcee Hammer package on 
some hard exoplanet orbit fitting problems.

We are also working on a Gaussian process model of the posterior log-likelihood surface.
This uses ideas from experimental design and stochastic gradient descent in the space of
log-likelihood surfaces.
Experiments with a 2D highly non-Gaussian Rosenbrock model show that it identifies
the log-likelihood surface reasonably well in just 50 likelihood evaluations.


\subsection*{Evidence based model selection}
\subsection*{??}

\section*{The research group and funding level}
The three central faculty of this proposal are Goodman, Hogg, and O'Neil.
Current graduate student Foreman Mackey, who expects to finish this spring,
would continue as a postdoctoral fellow.
We plan for the final proposal to recruit three more tenured, tenure track, or
equivalent level members at other institutions.
Significant funding will go to an annual multi-disciplinary workshop and other
outreach activities.
We plan to hire six postdoctoral fellows and support six graduate students, in 
equal proportion in mathematics and astrophysics.
Collaborations within NYU will be facilitated by the Center for Data Science, which
this project has significant synergy with.

We estimate a final budget of approximately \$1.2M/year.

Our research and outreach activities will lead to much interdisciplinary knowledge transfer
as computational mathematicians learn the special features of problems from astrophysics
and astrophysicists learn new mathematical and computational methods.  
This interaction will be especially intense at the faculty and postdoc level.


\section*{Community involvement and outreach}
One of the distinctive features of our research is the way we interact and plan to interact
with the larger mathematical and astrophysics communities.
In addition to meetings within our group, we are planning several ways to reach out to
and involve the larger computational mathematics and astrophysics communities.
We will hold special sessions on Bayesian computation (?) at the annual ?? meetings.
We will organize minisymposia on computational problems in astrophysics at annual
SIAM meetings. 
We will organize longer one week workshops that bring together mathematicians, statisticians,
computer scientists, and astrophysicists not directly involved in the collaboration of this proposal.
We will invite members of these larger communities to contribute to and collaborate with 
software we support.









\end{document}
