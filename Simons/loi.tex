% Copyright 2014 the authors.  All rights reserved.

% ## style guide:
% - nothing higher level than \paragraph{}
% - line break between sentences (that means you, O'Neil)
% - capitalize Collaboration
% - products and brands are \project{foo}

\documentclass[12pt]{article}
\usepackage{fancyheadings, color, hyperref}

% hypertex insanity
  \definecolor{linkcolor}{rgb}{0,0,0.4}
  \hypersetup{
    colorlinks=true,        % false: boxed links; true: colored links
    linkcolor=linkcolor,    % color of internal links
    citecolor=linkcolor,    % color of links to bibliography
    filecolor=linkcolor,    % color of file links
    urlcolor=linkcolor      % color of external links
  }
% Margins and spaces
  \setlength{\oddsidemargin}{0in}
  \setlength{\topmargin}{0in}
  \setlength{\headsep}{0.20in}
  \setlength{\headheight}{0.25in}
  \setlength{\textheight}{9.00in}
  \addtolength{\topmargin}{-\headsep}
  \addtolength{\topmargin}{-\headheight}
  \setlength{\textwidth}{6.50in}
  \setlength{\parskip}{0.5ex}
% Headings and footing
  \renewcommand{\headrulewidth}{0pt}
  \pagestyle{fancy}
  \lhead{\textsf{Goodman, Hogg \& O'Neil: \textit{Making Astrophysical Inferences Tractable}}}
  \rhead{\textsf{\thepage}}
  \cfoot{}

\newcommand{\foreign}[1]{\textsl{#1}}
\newcommand{\etal}{\foreign{et~al.}}
\newcommand{\project}[1]{\textsl{#1}}

\newcommand{\hoggitem}{$\bullet$}

\begin{document}\sloppy\sloppypar\thispagestyle{empty}

\noindent
\textbf{Simons Collaboration: Making Precise Astrophysical Inferences Tractable}\\
\textit{Collaboration Director:} Jonathan Goodman (NYU)\\
\textit{Core PIs:} Jonathan Goodman, David W. Hogg, Mike O'Neil (NYU)
\bigskip

We propose a Simons Collaboration working at the intersection of
Applied Mathematics and Astrophysics.
The focus of the Collaboration would be on probabilistic inference
(think: Bayes), where astrophysics has benefitted from huge successes.
The thinking in the astrophysics community about the next generation of
projects, however, is currently limited by computational tractability:

\hoggitem~For many problems it is nearly impossible to write down (let
alone compute) a likelihood function.
\hoggitem~Even when there \emph{is} a likelihood function, execution
can be very slow, and it usually has to be executed enormous numbers
of times in some kind of Markov Chain Monte Carlo method for
inference.
\hoggitem~There are often enormous numbers of nuisance parameters or
even continuous functions of space and time that need to be
marginalized out.
\hoggitem~Propagtion of noise from the pixel level in detectors (where
it is well understood) to the scientific quantities of most importance
can involve exceedingly complex hierarchical structure.
\hoggitem~At the ``bottom'' of this hierarchical structure is usually
the telescope and camera (and possibly spectrograph) that took the
data in the first place; this is part of the physical process by which
the measurements were made and must be modeled along with all
astrophysical signals.
\hoggitem~Many of the signals we care about---here we focus on
extra-solar planets and cosmological large-scale structure---are
extremely small and can only be detected by principled simultaneous
analyses of large collections of data.

The Core Team of PIs already has a good record at developing ideas of
applied mathematics in astrophysics contexts that address and
ameliorate these issues in specific astrophysics settings.
The Core Team has also had good success in propagating those methods
in the astrophysics community---and a wider community of physical
scientists---through publications, pedagogical events, and
high-quality software tools.
Our Collaboration proposal would involve growing our team into
something of order eight PIs across several institutions, recruiting
in a very targeted way from the Applied Mathematics and Astrophysics
communities.

\paragraph{Our contributions}

ALL OF THIS IS WAY TOO LONG!!

Contemporary physics---and science in general---increasingly relies on
finding subtle but important signals in masses of noisy data.
In its work to date, the Core Team has been looking for two
kinds of important signals in large data sets:
One is extra-solar planet (exoplanet) signals in precise
radial-velocity and photometric measurements of stars.
Here the exoplanet signals are tiny; the planets of greatest interest
produce signals that are smaller in amplitude than typical
contributions of stochastic stellar variability (from surface
convection, helioseismology, and starspots).
Thousands of planets have been found despite this, but each (even
small) improvement in modeling either the exoplanetary systems or else
the stochastic variability of the stars has led to new discoveries and
better measurements.

We have a set of successful exoplanet projects along these lines:
\hoggitem~We have built a suite of novel MCMC samplers (Hou \etal,
2012; Foreman-Mackey \etal, 2013; Hou \etal, 2014) some of which are
based on a self-tuning (affine-invariant) ensemble proposal (Goodman
\& Weare, 2010).
The Foreman-Mackey code is currently getting a dozen citations per
month in the refereed literature in a wide range of disciplines.
Some of the Hou codes compute the fully marginalized likelihood, which
can inform decisions in probabilistic inferences.
\hoggitem~We have built and applied fast linear algebra for
hierarchical off-diagonal low-rank matrices for doing Gaussian Process
regression, marginalization, and likelihood function evaluation
(Ambikasaran \etal, 2014).
Our code that implements this can do million-by-million problems to
machine precision on standard laptops and is now being used by many
researchers in different fields and is generating not just downloads
but open-source pull requests from researchers working on github.
\hoggitem~Our methods made it possible to resolve some controversies
in the astronomical literature, including one about planet
multiplicity in a radial-velocity system (Hou \etal, 2014), and
another about the reality of some transits around a highly variable
giant star (Barclay \etal, 2014).
They will also make new exoplanet searches more sensitive.
\hoggitem~We produced the first fully probabilistic hierarchical
inference of the exoplanet population (Foreman-Mackey \etal, 2014)
based on an importance-sampling method we published a few years
earlier (Hogg \etal, 2010).
This hierarchical inference gave different (by a factor of seven for
the abundance of Earth-like planets) answers than previous inferences,
in part because it made far more responsible use of the data
(propagating noise), and in part because it made far less restrictive
assumptions.

The other kind of signal we have been working on is the weak-lensing
distortions to galaxies induced by large-scale structure in the
density (dark matter).
This signal is hard to find because the weak-lensing distortions of
galaxies are tiny (percent-level) while galaxies have very complex
morphologies.
\hoggitem~So far, our biggest contribution is to deliver a system that
can compute the likelihood of the \emph{pixels} of the read-out
imaging from the telescope in terms of the cosmological parameters
(Schneider \etal, in preparation).
This system works like our exoplanet work with importance sampling.
We used it to enter the GREAT3 weak lensing challenge, which is being
written up now (Mandelbaum \etal, in prep).
\hoggitem~The pixel-level component of this project involves image
modeling; we have developed an extremely general and flexible
framework for this and open-source code, \project{The~Tractor} (Lang
\etal, in prep), which is now being used in multiple cosmological
projects.

An important unifying component to all of this is camera modeling
(and even spectrographs are cameras).
In addition to \project{The~Tractor}, we have been doing some work in
completely data-driven camera modeling.
\hoggitem~With Rob Fergus (NYU) we are building computer-vision-based
models for complex astronomical imagers (Fergus \etal, 2014).
These models made it possible for the first time ever to perform
infrared spectroscopy of some young exoplanets (Oppenheimer \etal,
2014).
\hoggitem~With Bernhard Sch\"olkopf (MPI-IS) we are using ideas from
causal inference to separate intrinsic and spacecraft-induced
variability in NASA \project{Kepler} light-curves (Wang \etal, in
prep; Foreman-Mackey \etal, in prep).
These models make predictions for stars using other stars; the models
have enormous flexibility (many input stars and flexible functions of
those stars in the prediction step).
\hoggitem~We are trying to build a data-driven model of the
\project{Kepler} focal plane to improve photometry in the \project{K2}
mission, the post-\project{Kepler} mission in which the spacecraft
does not point as precisely (Montet \etal, in prep).

\paragraph{New Directions}

We have identified mathematical and astrophysical directions for
research in the Collaboration.
The details of what we propose would depend precisely on the PIs we
recruit (see below) and the synergies we identify among them.
Here, for example, are some ideas about directions we have identified
as clearly promising.

EACH OF THE FOLLOWING NEEDS TO GET DOWN TO LIKE TWO SENTENCES OR SO

\hoggitem~\textbf{New MCMC methods:}
We have several ideas for much better MCMC samplers.
One, which is partly developed, is an MCMC version of the Gauss Newton method for
nonlinear least squares.
It is common to fit parameters by optimizing least squares fits of nonlinear models to data.
Sophisticated Gauss Newton Marquardt algorithms use user developed software to evaluate the 
model and the Jacobian matrix of model sensitivities.
We have an MCMC algorithm that uses this derivative information to become affine invariant.
The proposals are Gaussian centered on the Gauss Newton point.
We have discovered a robust backoff strategy related to line search without which the 
algorithm is extremely bad, but with which far out-performs the Emcee Hammer package on 
some hard exoplanet orbit fitting problems.

We are also working on a Gaussian process model of the posterior log-likelihood surface.
This uses ideas from experimental design and stochastic gradient descent in the space of
log-likelihood surfaces.
Experiments with a 2D highly non-Gaussian Rosenbrock model show that it identifies
the log-likelihood surface reasonably well in just 50 likelihood evaluations.

\hoggitem~\textbf{Evidence-based model selection} GOODMAN!

\hoggitem~\textbf{New regimes for fast linear algebra} ONEIL!

\hoggitem~\textbf{Effective models of astronomical instruments} HOGG

\hoggitem~\textbf{Noise models for stochastic variable stars} HOGG

\hoggitem~\textbf{Hierarchical models of exoplanet populations} HOGG

\hoggitem~\textbf{Marginalization of the cosmological density field} HOGG

\paragraph{Collaboration personnel and budget}

The three Core PIs for this proposal are Goodman, Hogg, and O'Neil.
Current graduate student Foreman Mackey, who expects to finish this spring,
would---if we can retain him---continue as a postdoctoral fellow.
We plan for the full proposal to recruit three to five more tenured,
tenure track, or equivalent level members at other institutions.
Our current list of recruiting candidates includes Tom Hou (Caltech),
John Johnson (Harvard), Mario Juric (UW), Phil Marshall (KIPAC),
Bernhard Sch\"olkopf (MPI-IS), ADD NAMES HERE.
We would plan to hire eight postdoctoral fellows and support eight
graduate students, in equal proportion in mathematics and
astrophysics.
Collaborations within NYU will be facilitated by and leverage the NYU
Center for Data Science, with which this project has significant
synergy.
We estimate a final budget of approximately 2.1M USD per year,
depending on the number of PIs and their commitments.

\paragraph{Collaboration plan and community involvement}

Our research and outreach activities will lead to much
interdisciplinary knowledge transfer as computational mathematicians
learn the special features of problems from astrophysics and
astrophysicists learn new mathematical and computational methods.
Within the Collaboration, this means quality time spent translating
problems, methods, terminology, and notation between fields, and a lot
of shared knowledge.
We would expect to leverage the technology of the NYU Center for Data
Science to hold regular inter-institutional group meetings to discuss
work in progress.

Externally, one of the distinctive features of our research is the way
we interact and plan to interact with the larger mathematical and
astrophysics communities.
In addition to an annual Collaboration meeting in New York at the
Simons Foundation---where we would invite not just our team but
relevant outsiders---we expect to engage the broader community through
hack days and tutorials at professional meetings.
It is already the case that some of the impact of our MCMC and
Gaussian Process code bases has been generated by our work operating Hack
Days at the American Astronomical Society (AAS) annual meetings and the
dotastronomy meetings for astronomy engineering, and also the annual
Astro Hack Week we have just started with collaborators at UW and
Berkeley.

We would propose to add events to the annual winter AAS meetings
during the Collaboration funding period, and do the same at the SIAM
and possibly JSM.
These would probably take the form of Applied-Math-meets-Astronomy
Hack Days, since these have been so successful previously, but the
style and scope would be related to the interests and capabilities of
the Collaboration faculty and employees.

Importantly, we intend for all work in the Collaboration to proceed
along a fully ``open'' model, as the work of the Core Team already
exemplifies.
We expect that a significant part of our community involvement will
proceed through sharing of code and projects in progress.
In particular, all the publications from the Collaboration would be
supported by open-source code with good licensing (MIT or similar).

\end{document}
