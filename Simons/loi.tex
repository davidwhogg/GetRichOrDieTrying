% Copyright 2014 the authors.  All rights reserved.

% ## style guide:
% - nothing higher level than \paragraph{}
% - line break between sentences (that means you, O'Neil)
% - capitalize Collaboration
% - products and brands are \project{foo}

\documentclass[12pt]{article}
\usepackage{fancyheadings, color, hyperref}
\usepackage{tabto}
% hypertex insanity
  \definecolor{linkcolor}{rgb}{0,0,0.4}
  \hypersetup{
    colorlinks=true,        % false: boxed links; true: colored links
    linkcolor=linkcolor,    % color of internal links
    citecolor=linkcolor,    % color of links to bibliography
    filecolor=linkcolor,    % color of file links
    urlcolor=linkcolor      % color of external links
  }
% Margins and spaces
  \setlength{\oddsidemargin}{0in}
  \setlength{\topmargin}{0in}
  \setlength{\headsep}{0.20in}
  \setlength{\headheight}{0.25in}
  \setlength{\textheight}{9.00in}
  \addtolength{\topmargin}{-\headsep}
  \addtolength{\topmargin}{-\headheight}
  \setlength{\textwidth}{6.50in}
  \setlength{\parskip}{0.5ex}
% Headings and footing
  \renewcommand{\headrulewidth}{0pt}
  \pagestyle{fancy}
  \lhead{\textsf{Goodman, Hogg \& O'Neil: \textit{Making Astrophysical Inferences Tractable}}}
  \rhead{\textsf{\thepage}}
  \cfoot{}

%  \usepackage{mathpazo}
%    \usepackage{pdfpages}

\newcommand{\foreign}[1]{\textsl{#1}}
\newcommand{\etal}{\foreign{et~al.}}
\newcommand{\project}[1]{\textsl{#1}}

\newcommand{\hoggitem}{$\bullet$}
\newcommand{\bioindent}{\tabto{1.5in}}

\usepackage[margin=1in]{geometry}
\usepackage{url}

\begin{document}\sloppy\sloppypar\thispagestyle{empty}

\begin{center}
\textbf{Simons Collaboration: Making Precise Astrophysical Inferences Tractable}\\
\textit{Collaboration Director:} Jonathan Goodman (Mathematics, NYU)\\
\textit{Core PIs:} Jonathan Goodman, David W. Hogg (Physics, NYU), Mike O'Neil (Math, NYU)
% \textit{Collaborating faculty:} Tom Hou (Applied Mathematics, Caltech)
\end{center}
\smallskip

We propose a Simons Collaboration to develop new mathematical and computational 
methods for probabilistic inference and discovery in astrophysics.
Probabilistic inference was crucial for many recent discoveries,
including the unexpected abundance and distribution of
exoplanets found in the NASA \project{Kepler} satellite data.
The next steps involve systematic simultaneous hierarchical inference
and more detailed physical models, both of the objects being studied and of
the instruments (lenses, CCD arrays, etc.).
These steps are infeasible with present algorithms.

This Letter of Intent identifies mathematical research areas that may enable
more subtle and powerful inference from astrophysical data,
and astrophysics research areas that may benefit from new math technology.
We focus on Earth-analog exoplanets and the distribution of dark matter,
but the application areas are broad.
The Core PIs' recent contributions demonstrate that this collaboration
between applied mathematics and astrophysics
can lead both to interesting new mathematics and to astronomical discoveries.
We mention the fast algorithms that were needed to evaluate determinants in
Gaussian process models, and systematic hierarchical evaluation of the evidence
integral used to decide between different planet models; these permitted
us to answer statistical questions that other groups have been unable to answer.

The computational cost of large-scale probabilistic modeling and Bayesian 
inference depends on the cost of a likelihood evaluation and the number of evaluations required.
The likelihood calls may involve large dense but structured linear algebra
that is amenable to fast methods, or they might involve differential 
equations with fast solvers, or they may have low accuracy but cheap approximations.
Better MCMC strategies require fewer likelihood evaluations per independent sample.
This may involve affine invariance, or learning from rejected samples
via response surface modeling, or using derivative information and
multi-level sampling.

Better algorithms will allow, and will be driven by, more complex, physical, and
detailed models for inference.
This includes more realistic noise and distortion models (begone, iid)
that take into account non-linear physics of the source
and the instrument.
It also includes simultaneous complex inference ranging from groups of stars to
populations of planets to the matter distribution of the universe.

Our proposal will include creative ways to reach out to and involve people in
the math and astrophysics communities,
through publications, pedagogical events (hack days and tutorials), and high-quality open-source
collaborative software.
Some of our success of these kinds are described below.
We will grow our team to six or eight PIs across several institutions, 
recruiting from the applied mathematics and astrophysics communities.
People listed below have agreed to participate.

\paragraph{Our contributions.}

Much of the research of the core PIs is in the directions described above.
Our mathematics/astrophysics collaborations, driven by exoplanet questions,
have been productive.
We created a suite of novel MCMC samplers 
(Hou \etal, 2012; Foreman-Mackey \etal, 2013; Hou \etal,
2014), some of which are based on a self-tuning (affine-invariant)
ensemble proposal (Goodman \& Weare, 2010).
We produced the first reliable (quantitatively accurate computationally 
verified error bars) computations of certain fully
marginalized likelihoods, which allowed us to address a controversy
about the number of planets about the star Gliese 581 (Hou \etal, 2014).
An MCMC code written by Foreman-Mackey receives a dozen
citations per month across a wide range of disciplines.

A project spearheaded by O'Neil and Hogg resulted in
fast (nearly-optimal), high-accuracy linear algebra
routines for calculating the inverse and determinant of the large-scale
dense covariance matrices which appear in Gaussian process modeling
(Ambikasaran \etal, 2014).
The resulting code can invert million-by-million matrices
accurately on laptops and is now being used by
several researchers in varying fields.
It generates downloads and also pull requests (contributions) 
on GitHub.
These tools allowed us to address a controversy concerning properties 
of some transits around highly variable stars (Barclay \etal, 2014).
%They will also make new exoplanet searches more sensitive.
We also produced the first fully probabilistic hierarchical inference of
the exoplanet population (Foreman-Mackey \etal, 2014) based on an
earlier importance-sampling method of ours (Hogg \etal, 2010).
This hierarchical inference yielded different (by a factor of seven for
the abundance of Earth-like planets) answers than previous inferences,
due in part to far more responsible use of the data (propagating noise)
and far less restrictive assumptions.

Hogg and collaborators have been using these and other tools to study
weak-lensing distortions ($\sim 0.1\%$), which are hard to find
due to the complex morphologies of galaxies.
The biggest contribution so far is a system
that can compute the likelihood of the \emph{pixels} of the read-out
imaging from the telescope in terms of the cosmological parameters
(Schneider \etal, in prep.).
% Entering a contest (results not announced), more "in preparation"
The pixel-level component of this project involves image modeling,
where we have developed a general and flexible framework.
Our open-source code, \project{The~Tractor} (Lang \etal,
in prep.) is now being used for cosmological projects by several other groups (??).

We have been investigating other camera models.
With Rob Fergus (NYU), we are building
computer-vision-based \emph{data-driven} models for complex astronomical 
imagers (Fergus \etal, 2014).
These made it possible to perform infrared spectroscopy of several 
young exoplanets (Oppenheimer \etal, 2014).
With Bernhard Sch\"olkopf (MPI-IS), we are working on multi-star models
of NASA \project{Kepler} light-curves using ideas from
causal inference. 
By modeling many star fluctuations at once, we are able to separate
common fluctuations due to spacecraft-induced variability from idiosyncratic
star fluctuations that contain transit information (Wang \etal, in prep.;
Foreman-Mackey \etal, in prep.).
We have begun to apply these ideas and techniques to 
possibly improve photometry in \project{K2}, the
post-\project{Kepler} mission in which the spacecraft is unable to point
as precisely as in the original design (Montet \etal, in prep).

\paragraph{New Directions}

We have identified many new mathematical and astrophysical avenues for
research in the Collaboration.
The details of what we finally propose will depend precisely on the
PIs that we recruit (see below) and the joint-abilities that we
identify among them.
Here, for example, are several directions that we have already
identified as clearly promising.

%\textbf{Effective models of astronomical instruments:}
%If two sources imaged in a telescope vary or are distorted in a covariant
%way, when the sources are physically separated by enormous distances
%(as are stars or galaxies in typical images, for example), then the covariant
%variability must be telescope-induced.
%This causal structure is a handle with which variability can be separated
%into intrinsic (astrophysical) and artificial (instrument-induced) components.
%Causal structure maps onto conditional dependences in hierarchical modeling;
%we expect large gains to be made possible for future time-domain astrophysics projects
%by capitalizing on this.

\textbf{Effective models of astronomical instruments:}
Simultaneous modeling of many or all images in a telescope image 
may allow us to distinguish variability due to the instrument from
variability of the objects themselves.  
This should make it possible to identify weak photometric or spectroscopic
signals with much more confidence.

\textbf{First principles camera simulation:} 
There has been much work aimed at accurate forward simulation of the optics 
of complex telescopes.
We bring to this endeavor fast numerical algorithms for Maxwell's equations,
high-frequency asymptotics, and the ability to calibrate complex systems 
to data using Bayesian Monte Carlo.


%\textbf{First principles camera simulation:} 
%The large lenses contained in many telescopes are sometimes millions of
%wavelengths in diameter, and therefore cannot be directly modeled using
%standard methods in computational electromagnetics.
%While geometric optics and diffraction have proven extremely useful over
%the years, many simulations of speckle??? and point-spread functions are
%still out of reach.
%The combination of moden fast numerical algorithms and recent results in
%high-frequency asymptotics should prove useful in constructing accurate
%forward simulations of telescopic hardware.  

\textbf{Noise models for stochastic variable stars:}
Signals from stars fluctuate because of asteroseismology 
(damped, driven normal modes), stochastic surface convection, star
spots, etc.
A Bayesian noise model built on this physics will make it possible to 
find ever smaller exoplanet signals with confidence.
Normal-mode noise translates directly into Gaussian processes 
for which we have very fast code.
In the star-spot and convection models, it also appears that there may
be stationary, non-trivial processes that can be developed and made
tractable.

%\textbf{Noise models for stochastic variable stars:}
%Stars vary stochastically because of surface convection,
%asteroseismology (damped, driven normal modes), and star spots.
%In all three cases, we see opportunities to build physically motivated
%``noise models'' that account for the time-coherence properties of
%these sources of variability.
%In the normal-mode case, we can literally model the stellar surface as
%a damped, driven oscillator driven by some noise spectrum; this
%translates directly into a Gaussian Process for which we have very
%fast code.
%In the star-spot and convection models, it also appears that there may
%be stationary, non-trivial processes that can be developed and made
%tractable.


\textbf{New MCMC methods:}
We have ideas for improved MCMC samplers.
One, which is partly developed, is an MCMC version of the Gauss-Newton
method for nonlinear least squares.
It uses derivative information to become affine invariant.
We have stochastic step size control algorithm that satisfies
detailed balance.
The theory of affine invariant samplers is difficult because they
lack compactness properties of traditional $O(n)$ invariant Metropolis
samplers.
There are hints that better affine invariant samplers depend on 
controlling the different ways metrics can detegenerate.


%It is common to fit parameters by optimizing least squares fits of
%nonlinear models to data.  Sophisticated Gauss Newton Marquardt
%algorithms use user developed software to evaluate the model and the
%Jacobian matrix of model sensitivities.  We have an MCMC algorithm that
%uses this derivative information to become affine invariant.  The
%proposals are Gaussian centered on the Gauss Newton point.  We have
%discovered a robust backoff strategy related to line search without
%which the algorithm is extremely bad, but with which far out-performs
%the Emcee Hammer package on some hard exoplanet orbit fitting problems.

%  A whole paragraph deleted!  OUCH!

\textbf{Posterior likelihood modeling:}
Using ideas from experimental design and response surfaces, we plan to 
model the posterior log-likelihood surface using
Gaussian Processes.
Numerical experiments show that our technique is able to
identify some highly non-Gaussian models two-dimensions in as few as
fifty likelihood evaluations.

\textbf{Novel analysis-based methods for dense linear algebra:}
The Core Team has already succeeded in producing algorithms for Gaussian
Processes which scale nearly-linearly in the dimension of the likelihood
function.
Further improvements to this algorithm can be made, namely optimizations
for distributed computing environments, improvements in the accuracy
which rely on the analytic formulation of the covariance kernel, and
extensions to hierarchical probabilistic structures.

\textbf{Hierarchical models of exoplanet populations:}
The Core-Team work on making hierarchical problems tractable has
only scratched the surface of what might be possible going forward.
It is now (in part because of the Core Team's influence) becoming
understood in the astrophysics community that hierarchical inference
is a requirement to propagate catalog-level or pixel-level noise to
population-level inferences.
Good samplers for hierarchical models would be widely applicable in
and outside of astrophysics.

%\textbf{Hierarchical models of exoplanet populations:}
%The Core-Team work on making hierarchical problems tractable, by
%sampling with interim priors in the data center, and then combining
%those samplings using importance sampling in the final inference, has
%only scratched the surface of what might be possible going forward.
%It is now (in part because of the Core Team's influence) becoming
%understood in the astrophysics community that hierarchical inference
%is a requirement to propagate catalog-level or pixel-level noise to
%population-level inferences.
%There is abundant work to be done applying and extending these
%methods, in new areas of astrophysics, but also by exploring new ideas
%in MCMC sampling, where the samplers can capitalize on the
%hierarchical structure.


\textbf{Marginalization of the cosmological density field:} 
It appears possible now to create flexible representations of the
two-point autocorrelation function relevant to cosmology, including
the general large-scale structure covariance and also the baryon
acoustic feature (Eisenstein \etal, 2005).
Cosmological inferences, which currently seek to estimate the two-point 
function, could instead infer and marginalize over the density field itself.
These could be based on Bayesian priors using flexible representations of the
two-point function relevant to cosmology, including
the general large-scale structure covariance and also the baryon
acoustic feature (Eisenstein \etal, 2005).
Direct field models allow more realistic noise modeling and it would not rely
on assumptions of Gaussianity, which may be unwarranted.
Such massive inference is impractical without new computational methods
that are a goal of this Collaboration.



%\textbf{Marginalization of the cosmological density field:} 
%It appears possible now to create flexible representations of the
%two-point autocorrelation function relevant to cosmology, including
%the general large-scale structure covariance and also the baryon
%acoustic feature (Eisenstein \etal, 2005).
%In principle, cosmological inferences (which currently proceed by
%comparing point estimates of the two-point function) could proceed
%through proper inference and marginalization of the density field.
%Several methods from numerical approximation theory and applied analysis
%seem to indicate that this can be made into a computationally tractable 
%problem.
%A flexible parameterization of correlation functions, and their use in
%fast marginalizations over random fields, will surely have wide-ranging
%applications in various other fields.

\paragraph{Collaboration personnel and budget}

The three Core PIs for this proposal are Goodman, Hogg, and O'Neil.
Current graduate student Foreman-Mackey (NYU), who expects to finish
this spring, will---if we can retain him---continue as a postdoctoral
fellow.
Tom Hou (Caltech) has agreed to join the collaboration.
He is a pioneer in multi-level sampling methods.
We plan for the full proposal to recruit two to four more tenured,
tenure track, or equivalent level members at other institutions.  Our
current list of recruiting candidates includes: John
Johnson (Harvard), Mario Juric (UW), Phil Marshall (KIPAC), and Bernhard
Sch\"olkopf (MPI-IS).
We will plan to hire eight postdoctoral fellows and support eight
graduate students, in equal proportion in mathematics and astrophysics.
Collaborations within NYU will leverage the NYU Center for Data Science,
with which this project has significant relevance.
We estimate a final budget of approximately 2.1M USD per year,
based on a total of eight PIs and their commitments approximate.

\paragraph{Collaboration plan and community involvement}

Our research and outreach activities will lead to a mutually beneficial
transfer of knowledge from computational mathematicians to
astrophysicists, and vice versa.
The mathematicians on our team will learn the special features of
problems from astrophysics, and astrophysicists will be exposed to new
mathematical and computational methods.
Within the Collaboration, this means quality time spent translating
problems, methods, terminology, and notation between fields.

Externally, one of the distinctive features of our research is the way
in which we interact, and plan to interact, with the larger mathematical
and astrophysics communities.
In addition to an annual Collaboration meeting in New York at the Simons
Foundation---where we would invite not just our team but relevant
outsiders---we expect to engage the broader community through \emph{Hack
Days} and tutorials at professional meetings, especially the AAS and the SIAM meetings.
This approach has already proven to be extremely successful with regard
to the impact of our MCMC and Gaussian Process software packages.

Equally important, we intend for all work in the Collaboration to
proceed along a fully \emph{open} model, as the work of the Core Team
already exemplifies.
We expect that a significant part of our community involvement will
proceed through the sharing of software packages, as well as projects in
progress.
In particular, all the publications from the Collaboration would be
supported by open-source code with liberal licensing (MIT or similar).


\end{document}
