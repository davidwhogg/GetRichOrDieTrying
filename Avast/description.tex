% This file is part of the GetRichOrDieTrying project.
% Copyright 2017 David W. Hogg

% # Style notes
% - Third person, not first person!!

% # To-do list
% - Write zeroth draft!
% - Put in references.
% - Need to call down from the laundry list and up from the proposed activities.
% - Replace highlighting with something non-ugly.

\documentclass[12pt, fullpage, letterpaper]{article}
\usepackage{fancyheadings, graphicx, soul}
\setlength{\headsep}{2ex}
\input{hogg_nsf}
  \renewcommand{\headrulewidth}{0pt}
  \pagestyle{fancy}
  \lhead{\textsf{David W. Hogg / Data-driven approaches for extreme precision radial-velocity measurement}}
  \rhead{\textsf{\thepage}}
  \cfoot{}

\begin{document}

\noindent
From a fundamental perspective, there is essentially no limit to how
precisely astronomers might measure the radial velocity of the center of mass
of a distant star with an extreme precision radial-velocity (\EPRV) spectrograph.
Of course in real life, there are real limits, set
by (for example) photon noise, stellar surface convection and activity,
variations in the atmosphere, and spectrograph flexure and illumination. In practice,
essentially no stars have had their center-of-mass velocities measured
consistently with an empirical scatter (\acronym{RMS} around the mean,
say) smaller than $1\,\mps$.

At the same time, \EPRV\ is responsible for hundreds of \foreign{ab initio}
exoplanet discoveries and many hundreds more \foreign{a posteriori}
characterizations of planets
discovered elsewhere. That is, \EPRV\ is hugely successful.
Because most of the discoveries and characterizations have been made
at or near precision limits---that is, near what is detectable at noise levels
greater than $1\,\mps$ per epoch---\hl{any improvement in precision directly
translates into increased scientific return on investment},
both from existing archival data and from new data from new programs.

This is promising! Most of the noise sources \emph{other} than photon noise
are generated by macroscopic physical processes that are amenable to both
physical and statistical modeling and prediction. And there are enormous amounts
of data in hand, and far more to come. That opens up the hope
that data-driven models can be applied to these nuisances and directly improve
\EPRV\ precision (and therefore capabilities). \hl{This proposal is to quantify
these nuisances, develop a justifiable plan for addressing the tallest
(addressable) poles first, and then building models that address those tall
poles.}

This proposal begins with a context-setting discussion of the sources
of noise in \EPRV\ experiments---where we are being very liberal in the definition of the
word ``noise''---followed by a set of proposed activities. The activities fall into two
separate \emph{Components}: Component~1 is about understanding the relative
(and absolute) contributions of different kinds of noise. Component~2
is about building models of noise sources---nuisances---that will
deliver improved precision for future experiments and especially \NASA's
\NNEXPLORE\ program. Each of the two Components of the proposed activities
further subdivide into subcomponents of varying levels of specificity.

\paragraph{What are the sources of radial-velocity noise?}
There are many possible motivations for precisely monitoring the radial
velocity of a star.
If, however, the goal is the detection and characterization of exoplanets,
then many of the physical effects that put radial-velocity imprints into
a spectrogram are nuisances, or confusing.
These nuisances will be referred to as ``noise'' in what follows.

One comment: The goal of \EPRV\ is to measure the \emph{center of mass} of
the target star.
It is this center-of-mass motion that reflects the exoplanetary system motions.
Some of the noise sources discussed here affect the velocity of the star,
but the velocity of the \emph{surface} of the star. That is, they mask the
motion of the center of mass, which is our target.

The sources of noise that follow are listed in a \emph{causal order} along the
line of sight from the star to the data analyst.

\paragraph{Surface convection:}
The surface of the Sun is roiling in high-Rayleigh-number convection,
with hundreds of thousands or millions of convective cells turning
over chaotically on the surface. Each of these cells---the granulation---has internal
velocity shear in the $\kmps$ range, so averaging over many cells,
there could be up to $\mps$ noise from convection alone in the
\EPRV\ data (CITE papers Bedell sent).
(Discussion in the literature of convection noise is often somewhat
confused with activity, to be discussed below.)

One interesting thing about convection is that there is a convective
blue-shift effect (CITE) in which the hotter, up-welling material is
(by dint of hotter) is brighter than the down-welling material, and leads
to a net observed outward velocity of the Sun's surface!
This effect was discovered long ago as a puzzle.
The convective blue-shift is irrelevant to the present discussion---in
\EPRV\ we only care about velocity \emph{changes}---but it highlights
the possibility that the convection could be analyzed spectrally and
corrected for or modeled simultaneously in an \EPRV\ experiment. We
will propose doing this, below.

In addition to possible temperature signatures of convection, there
are characteristic time-scales of convective overturn, and so the convection
pattern changes with time in a continuous way that can (at least in principle)
be modeled, with either a data-driven statistical model or a physics-driven
correlation model.
Stellar surfaces also rotate (differentially); this rotation both affects
the convection patterns, and also rotates different stellar surface patches
into view on the same time-scales as the convection itself changes.

\paragraph{Asteroseismic modes:}
The \HARPS\ instrument has been used to directly measure asteroseismic
P-modes in the radial-velocity signal (CITE).
That is, the asteroseismic modes don't
just lead to brightness variations; they also lead to surface radial-velocity
variations.
For Sun-like stars, these asteroseismic modes have periods on the order of minutes,
so these are mitigated by taking long-ish exposures, longer than five minutes, say.
However, this mitigation is not complete: There are many modes over a range of periods,
and they have beat frequencies that have much longer periods than any of the fundamental
modes.
That is, the asteroseismic modes leave residual contributions in any finite observation.

The nice thing about these modes is that absolutely everything about their statistics
is calculable.
The characteristic frequencies, the frequency differences, and the mode amplitudes are
all very smooth functions of stellar parameters.
Therefore it is possible to estimate the effect of asteroseismic modes on any well-defined
observing program, and even to fit a detailed stochastic model to any data stream.

This proposal is not to perform further work along these lines, but that is in part because
the full modeling of asteroseismic modes under finite integration will be solved
by advances happening now in Gaussian Processes (CITE DFM).

\paragraph{Activity and sunspots:}
Show results we have in hand now.

Plages may suppress convection locally. So change the net convective signal.

Interplay of spots and rotation. But rotations aren't always visible.

latitude dependence to sunspots of possible relevance.

\paragraph{Planetary system and companion noise:}
Many stars---perhaps literally all stars---host some
kind of planetary system.
Many of these include close-in planets and probably asteroids, comets,
and zodiacal dust.
There are many modes by which these extra-solar systems can put
confusing or noisy imprints on \EPRV\ data:

For example, a complex, massive planetary system might introduce
radial-velocity variations that look like noise, despite being
produced by coherent orbits.
For another, a close-in planet or dust disk (or alien
megastructure) can produce reflected light that pollutes the stellar
spectrum and distorts the shapes and positions of absorption lines,
possibly in a complex way (if the reflection is spectrally
non-trivial).
For another, there might be comets, asteroids, or non-axisymmetric
dust that lead to time-dependent reflections or emission.

In general and in principle there could be ``adversarial'' planetary
systems that are difficult to distinguish from observational noise or
jitter.
For example,
if at small semi-major axes planetary systems have any tendancy---or
any subset of them have any tendancy---to have dense orbital systems
with planetary mass scaling like the square-root of semi-major axis,
the radial-velocity reaction of the host star might be indistinguishable
from white noise on short time scales.

\paragraph{Interstellar lines:}
There are diffuse interstellar bands, and atomic and molecular lines
that are imprinted on stellar spectra from intervening interstallar
medium between the star and us.
If there are reasonable relative velocities between the star and the
interstellar medium (which are generic), these interstallar lines will
change shape and velocity over time.
At $20\,\kmps$, the interstellar medium crosses the surface of a star
in 20 hours or so. If the medium is turbulent on small scales, or contains
jets or flows, this
could lead to substantial changes from epoch to epoch.

Of course the strong interstellar lines are known and can be modeled or masked
to remove their influence on \EPRV\ measurements.
However, there could be large families of very weak
lines---``micro-DIBS'' to coin a phrase---that could be time-variable
and distorting the absorption lines on which \EPRV\ rests.
Some lines (for example some Ca lines) exist both in stellar photospheres
and the interstellar medium. These lines are particularly dangerous.
Again, these can be masked or modeled in principle. But there might be
larger families of such lines than are currently known.
Finding and modeling these will be part of the proposed project below.

\paragraph{Cosmic rays:}
Along with interstellar medium, interstellar space is filled with cosmic
rays.
These don't affect the stellar spectrum directly, but they do hit the detector
and distort the two-dimensional spectrum.
Right now these probably don't cause any radial-velocity harm.
However, low-amplitude or un-masked cosmic rays might affect spectral extraction,
continuum normalization, or individual lines.

\paragraph{Atmosphere and tellurics and micro-tellurics:}
There are enormous numbers of telluric lines in the published spectral atlas
of the Sun (CITE). However, it has been proposed that there might be enormous
numbers of additional micro-tellurics, too faint to be cataloged but not too faint
to be messing with \EPRV\ measurements (CITE).
The nice thing about telluric components to the spectrum is that they must
have velocities that are consistent with atmospheric velocity structure,
observatory location,  and telescope pointing.
Furthermore, any model of micro-tellurics can be tied to the macro-tellurics
in velocity.

In what follows, this proposal is (in part) to explore the possibility that the tellurics can be
explained with a low-dimensional model.
Telluric absorption is not constant with time: There are water,
CO$_2$, and dust (and other pollutant) concentrations that vary with time, and
there are temperature and pressure variations with time and altitude.
However, it is thought that the number of controlling variables (latent variables or
factors) must be relatively small, while the number of observed wavelengths is
extremely large.
This opens up the possibility of creating a fully data-driven but extremely accurate
model of the telluric absorption. This is proposed below.

Similar to, but unrelated (strictly) to tellurics are differential
refraction, and Sun and Moon scattering into the fiber.
All of these effects are atmosphere-dominated and might distort the spectrum
in ways that lead to velocity shifts that matter at \EPRV\ precision.

\paragraph{Telescope optics}
Flexure, and fiber. The position of the star on the fiber, and the
incomplete ``scrambling'' in the fiber.

\paragraph{Spectrograph calibration:}
Accuracy and precision of the wavelength solution.

Combs and gas cells.

If lamps, long-term changes in the lamps.

LSF as a function of wavelength.

Hierarchical self-calibration---living the dream.

\paragraph{Photon noise:}

\paragraph{Data extraction and analysis:}
Two-d to one-d extraction and distortions to line shapes?

CRs appear again here.

Internal reflections of the spectrum inside the spectrograph.

CCV has to be stable to movements of the spectrum relative to the
pixels. Edge effects, and flat-field effects, other detector effects.

CCV should be using accurate spectral and telluric models.

Barycentric corrections.

Working entirely in 2-d---living the dream. Some systematics will be
simpler or more apparent in 2-d than the individual 1-d orders.

\paragraph{Why this PI? Why now?}
Here we discuss successes with previous projects.

And here we also discuss \NNEXPLORE, and the new context for
\NASA\ Exoplanet Research.

Few open-source pipelines. MB only knows of one.

No open-source pipelines get down to m/s precision.

\paragraph{Component 1---A quantitative noise budget for the \HARPS\ data:}
A set of procedures for assessing the 

\paragraph{Component 1a---Cross-correlation radial velocities:}
Comments on the relationship to max-like. How things go wrong
if the templates are bad, or if there are unmodeled components?
Possibility that some of the noise budget comes from these
problems. Cosmic rays. Tellurics. Etc.

\paragraph{Component 1b---Something else:}

\paragraph{Component 1c---Causal inference on the noise budget:}

\paragraph{Component 1x---Other projects:}
Interpolation onto a common grid / combined spectrum.

\paragraph{Component 2---Reducing noise by modeling nuisances:}

\paragraph{Component 2a---A non-parametric model for an observed spectrum:}
Modeling spectrum, tellurics, and ISM all with some kind of
process (probably Gaussian process).

Make remarks about very fast linear algebra.

\paragraph{Component 2b---A finite-dimensionality tellurics and interstellar model:}
How we can look at the variability in the tellurics.
How we can determine the principal directions of variation.
How we can capitalize on these results.

Emphasize that we can use all stars ever observed with \HARPS\ to
build and support this model.

\paragraph{Component 2c---Spectral signatures of convection (and other nuisances):}
Convection is not activity! But open with wavelength-dependence of
activity and activity indicators.

\paragraph{Component 2d---Hierarchical models for the time domain:}
Ultimately, we have very strong prior beliefs about some aspects
of the radial-velocity variability. For examples: The orbital
companions imprint Kepler-like variations. The asteroseismic modes
imprint combs of quasi-periodic signals. The convection is stochastic
but with identifiable time covariance.

\paragraph{Component 2x---Other possible projects:}
Other projects. Laundry list.

\paragraph{Prior \NASA\ support:}
The \GALEX\ and \Ktwo\ projects are of great relevance.

\paragraph{Management and timeline:}
Very lean project; just the PI, unfunded collaborators, and graduate students.

\end{document}
