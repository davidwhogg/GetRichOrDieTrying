\documentclass[12pt, fullpage, letterpaper]{article}
\usepackage{fancyheadings, graphicx}
\setlength{\headsep}{2ex}
\input{hogg_nsf}
  \renewcommand{\headrulewidth}{0pt}
  \pagestyle{fancy}
  \lhead{\textsf{David W. Hogg / Data-driven approaches for extreme precision radial-velocity measurement}}
  \rhead{\textsf{\thepage}}
  \cfoot{}

\begin{document}

\paragraph{What are the sources of radial-velocity noise?}
Laundry list it. Refer to literature. Discuss the basis on
which the quantitative estimates have been made.

\paragraph{A quantitative noise budget for the \HARPS\ data}
A set of procedures for assessing the 

\paragraph{Cross-correlation radial velocities}
Comments on the relationship to max-like. How things go wrong
if the templates are bad, or if there are unmodeled components?
Possibility that some of the noise budget comes from these
problems.

\paragraph{A non-parametric model for an observed spectrum}
Modeling spectrum, tellurics, and ISM all with some kind of
process (probably Gaussian process).

Make remarks about very fast linear algebra.

\paragraph{A finite-dimensionality tellurics model}
How we can look at the variability in the tellurics.
How we can determine the principal directions of variation.
How we can capitalize on these results.

Emphasize that we can use all stars ever observed with \HARPS\ to
build and support this model.

\paragraph{Hierarchical models for the time domain}
Ultimately, we have very strong prior beliefs about some aspects
of the radial-velocity variability. For examples: The orbital
companions imprint Kepler-like variations. The asteroseismic modes
imprint combs of quasi-periodic signals. The convection is stochastic
but with identifiable time covariance.

\paragraph{Prior \NASA\ support}
The \GALEX\ and \Ktwo\ projects are of great relevance.

\paragraph{Management and timeline}
Very lean project; just the PI, unfunded collaborators, and graduate students.

\end{document}
