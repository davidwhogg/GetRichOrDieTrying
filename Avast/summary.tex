\documentclass[12pt, fullpage, letterpaper]{article}
\usepackage{fancyheadings, graphicx}
\setlength{\headsep}{2ex}
%----- exact 1-in margins
% NB: headheight and headsep MUST exist and be set
\setlength{\textwidth}{6.5in}
\setlength{\textheight}{9in}
\addtolength{\textheight}{-1.0\headheight}
\addtolength{\textheight}{-1.0\headsep}
\setlength{\topmargin}{0.0in}
\setlength{\oddsidemargin}{0.0in}
\setlength{\evensidemargin}{0.0in}

%----- typeset certain kinds of words
\newcommand{\observatory}[1]{\textsl{#1}}
\newcommand{\package}[1]{\textsf{#1}}
\newcommand{\project}[1]{\textsl{#1}}
\newcommand{\an}{\package{Astrometry.net}}
\newcommand{\boss}{\project{BOSS}}
\newcommand{\des}{\project{DES}}
\newcommand{\euclid}{\observatory{Euclid}}
\newcommand{\EA}{\observatory{Exoplanet Archive}}
\newcommand{\gaia}{\observatory{Gaia}}
\newcommand{\Gaia}{\gaia}
\newcommand{\galex}{\observatory{GALEX}}
\newcommand{\Kepler}{\observatory{Kepler}}
\newcommand{\kepler}{\Kepler}
\newcommand{\lsst}{\observatory{LSST}}
\newcommand{\MAST}{\observatory{MAST}}
\newcommand{\sdss}{\observatory{SDSS}}
\newcommand{\Spitzer}{\observatory{Spitzer}}
\newcommand{\spitzer}{\Spitzer}
\newcommand{\TESS}{\observatory{TESS}}
\newcommand{\thetractor}{\package{The~Tractor}}
\newcommand{\wfirst}{\observatory{WFIRST}}
\newcommand{\latin}[1]{\textit{#1}}
\newcommand{\eg}{\latin{e.g.}}
\newcommand{\etal}{\latin{et~al.}}
\newcommand{\etc}{\latin{etc.}}
\newcommand{\ie}{\latin{i.e.}}
\newcommand{\vs}{\latin{vs.}}

%----- math shih
\newcommand{\given}{\,|\,}

%----- typeset journals
% \newcommand{\aj}{Astron.\,J.}
% \newcommand{\apj}{Astrophys.\,J.}
% \newcommand{\apjl}{Astrophys.\,J.\,Lett.}
% \newcommand{\apjs}{Astrophys.\,J.\,Supp.\,Ser.}
% \newcommand{\mnras}{Mon.\,Not.\,Roy.\,Ast.\,Soc.}
% \newcommand{\aap}{Astron.\,\&~Astrophys.}

%----- Tighten up paragraphs and lists
\setlength{\parskip}{0.0ex}
\setlength{\parindent}{0.2in}
\renewenvironment{itemize}{\begin{list}{$\bullet$}{%
  \setlength{\topsep}{0.0ex}%
  \setlength{\parsep}{0.0ex}%
  \setlength{\partopsep}{0.0ex}%
  \setlength{\itemsep}{0.0ex}%
  \setlength{\leftmargin}{1.0\parindent}}}{\end{list}}
\newcounter{actr}
\renewenvironment{enumerate}{\begin{list}{\scriptsize{\textbf{\textsf{(\arabic{actr})}}}}{%
  \usecounter{actr}%
  \setlength{\topsep}{0.0ex}%
  \setlength{\parsep}{0.0ex}%
  \setlength{\partopsep}{0.0ex}%
  \setlength{\itemsep}{0.0ex}%
  \setlength{\leftmargin}{1.0\parindent}}}{\end{list}}

%----- mess with paragraph spacing!
\makeatletter
\renewcommand\paragraph{\@startsection{paragraph}{4}{\z@}%
                                    {1ex}%
                                    {-1em}%
                                    {\normalfont\normalsize\bfseries}}
\makeatother

%----- Special Hogg list for references
  \newcommand{\hogglist}{%
    \rightmargin=0in
    \leftmargin=0.25in
    \topsep=0ex
    \partopsep=0pt
    \itemsep=0ex
    \parsep=0pt
    \itemindent=-1.0\leftmargin
    \listparindent=\leftmargin
    \settowidth{\labelsep}{~}
    \usecounter{enumi}
  }

%----- side-to-side figure macro
%------- make numbers add up to 94%
 \newlength{\figurewidth}
 \newlength{\captionwidth}
 \newcommand{\ssfigure}[3]{%
   \setlength{\figurewidth}{#2\textwidth}
   \setlength{\captionwidth}{\textwidth}
   \addtolength{\captionwidth}{-\figurewidth}
   \addtolength{\captionwidth}{-0.02\figurewidth}
   \begin{figure}[htb]%
   \begin{tabular}{cc}%
     \begin{minipage}[c]{\figurewidth}%
       \resizebox{\figurewidth}{!}{\includegraphics{#1}}%
     \end{minipage} &%
     \begin{minipage}[c]{\captionwidth}%
       \textsf{\caption[]{\footnotesize {#3}}}%
     \end{minipage}%
   \end{tabular}%
   \end{figure}}

%----- top-bottom figure macro
 \newlength{\figureheight}
 \setlength{\figureheight}{0.75\textheight}
 \newcommand{\tbfigure}[2]{%
   \begin{figure}[htp]%
   \resizebox{\textwidth}{!}{\includegraphics{#1}}%
   \textsf{\caption[]{\footnotesize {#2}}}%
   \end{figure}}

%----- deal with pdf page-size stupidity
\special{papersize=8.5in,11in}
\setlength{\pdfpageheight}{\paperheight}
\setlength{\pdfpagewidth}{\paperwidth}

% no more bad lines!
\sloppy\sloppypar

  \renewcommand{\headrulewidth}{0pt}
  \pagestyle{empty}

\begin{document}

\paragraph{Data-driven approaches for extreme precision radial-velocity measurement}

In a forward-looking move, NASA has partnered with NSF to create
NN-EXPLORE, a project to provide extremely high precision ground-based
radial velocity measurements for exoplanet discovery and
characterization. The instrument NEID being built for this partnership
will be an extremely stable and precise spectrograph. In addition to
NEID, there are multiple existing and new projects world-wide to
provide large numbers of extreme precision radial velocities for the
follow-up of NASA-led exoplanet discoveries. Typically these
spectrographs produce internal precisions, or theoretically best-case
measurement noise, on the order of 0.1 m/s. This is precise enough to
detect and characterize Earth-like planets around Sun-like stars!

At the present, no spectrographs are returning stellar measurements
with empirical radial-velocity scatter below something like 1
m/s. There are many reasons for the mismatch between the theoretical
(0.1 m/s) and empirical (1 m/s) precisions: Spectrograph calibration
is non-trivial; there are spectral distortions from flexure and
atmospheric conditions; stars oscillate seismically; stellar surfaces
are heaving stochastically with convection; and there are spots and
time-dependent activity on the rotating stellar surface. Preliminary
work by my group suggests that convection noise may dominate these
effects for quiet, Sun-like stars, but this is by no means
demonstrated at the present day.

This project involves two components: A small causal inference
component, and a larger non-parametric modeling component. In the
causal inference component, we will make use of ideas from statistics
to separate the empirical variance in radial-velocity measurements
into different components with different causal origins. This involves
building flexible, data-driven models of covariances between
radial-velocity offsets and housekeeping data that are connected to
different causal inputs. The output of the causal modeling will be a
component separation of the noise sources, but also high-level
information about which systematics are most important to address with
further modeling.

In the non-parametric modeling component, we will make use of ideas
from probabilistic (Bayesian) machine learning to build flexible,
generative models of the signals that combine to produce noise or
scatter in the radial-velocity measurements. Some of our approaches
will be along the lines of building photon-noise-saturating empricial
models of mean stellar spectra (at relevant resolutions in the 100,000
range) and also low-dimensional models for telluric and interstellar
absorption. Other approaches will be along the lines of dimensionality
reduction and systematics modeling for wavelength solutions in
calibration meta-data. Other approaches will include building
wavelength-localized but time-stochastic models for stellar activity,
and quasi-periodic models for asteroseismic oscillations. Others will
use flexible models to find covariances between subtle spectral
indicators and convection asymmetries. In each causally distinct
component, the objective will be to use extremely flexible or
non-parametric models (like, for instance, stationary and
non-stationary Gaussian Processes), but also to produce proabilities
over data, or likelihood functions.  Our preliminary work in this
area, and our past work in transit photometry, suggest that these
approaches are likely to substantially improve each stage of exoplanet
measurement from spectrograph calibration to final orbit-fitting, for
a wide range of different spectrographs.

The budget will support graduate-student researchers at NYU and the
PI, plus travel, publications, and supplies. The deliverables will
include both methods and code, plus analyses (data-driven models and
searches) of currently available archival radial-velocity data. All
results will be published in fully reproducible publications in the
AAS Journals, and all code and data will be released publicly with
open licenses.

\end{document}
