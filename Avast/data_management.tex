\documentclass[12pt, fullpage, letterpaper]{article}
\usepackage{fancyheadings, graphicx}
\setlength{\headsep}{0ex}
\input{hogg_nsf}
  \renewcommand{\headrulewidth}{0pt}
  \pagestyle{empty}

\begin{document}

\paragraph{Data management plan:}
This project will not produce new data \foreign{per se}, but it does produce
methods, code, and some value-added meta-data for radial-velocity experiments.
All of the code, documentation, data, and all written documents relating to the
project will be maintained in publicly visible version-control repositories on
the GitHub platform. These repositories can be cloned and reproduced by any
user in the world.

The code and documents will be licensed for re-use by others under permissive
licenses (MIT for code, CC-by for documents). This permits re-use, reproducibility,
and re-publication by others.

Although we do not expect to deliver radial-velocity measurements directly, if
we do create any catalog of stellar radial velocities, we will return these to
the \HARPS\ Archive (or whatever is the relevant hosting archive). We will also
publish them in GitHub repositories, and license them for re-use.

Any cite-able data, code, or documentation will be delivered to Zenodo for
preservation and creation of a cite-able DOI.
All documents and papers will be submitted to the arXiv and also \AAS-family
journals for long-term preservation, cite-ability, and publication.

\end{document}
