\documentclass[12pt]{article}
\usepackage{fancyheadings}
\input{hogg_nsf}
  \renewcommand{\headrulewidth}{0pt}
  \pagestyle{fancy}
  \lhead{\textsf{David W. Hogg}}
  \chead{\textsf{New Probabilistic Methods for Observational Cosmology}}
  \rhead{\textsf{\thepage}}
  \cfoot{}
\begin{document}\sloppy\sloppypar

...The theory of experimental cosmology

...Standard practice in cosmology today:
For example, in typical large-scale structure surveys, the standard
method of imaging the sky, creating a catalog of sources, obtaining a
best value for each source's redshift, computing a standard point
estimate of the auto-correlation function, and then performing
cosmological inferences on that correlation function, is a daisy-chain
of lossy data-analysis steps.

...Why doing the right thing would be hard

...Example:  Using p(z) in standard large-scale structure!

...What have we achieved so far?

...In exoplanets

...In weak lensing

...How are we making these things possible?  Applied-math developments

...Very fast GPs

...Samplers that don't require tuning

...Importance sampling for hierarchical inference

...Aside:  $p(d\given z)$ vs $p(z\given d)$

...Pedagogical role in the community

\section*{Element 1: Marginalizing out density fields}

\section*{Element 2: Using and improving probabilistic redshift information}

\section*{Element 3: Using probabilistic shape and PSF information}

\section*{Prior NSF support}

...In addition to this NSF support, NASA support and Moore--Sloan.

...Moore--Sloan \emph{not} providing direct research support!

\section*{Broader impacts}

\section*{Project management}

\end{document}
