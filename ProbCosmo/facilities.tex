\documentclass[letterpaper,12pt]{article}
\setlength{\headheight}{0ex}
\setlength{\headsep}{0ex}
%----- exact 1-in margins
% NB: headheight and headsep MUST exist and be set
\setlength{\textwidth}{6.5in}
\setlength{\textheight}{9in}
\addtolength{\textheight}{-1.0\headheight}
\addtolength{\textheight}{-1.0\headsep}
\setlength{\topmargin}{0.0in}
\setlength{\oddsidemargin}{0.0in}
\setlength{\evensidemargin}{0.0in}

%----- typeset certain kinds of words
\newcommand{\observatory}[1]{\textsl{#1}}
\newcommand{\package}[1]{\textsf{#1}}
\newcommand{\project}[1]{\textsl{#1}}
\newcommand{\an}{\package{Astrometry.net}}
\newcommand{\boss}{\project{BOSS}}
\newcommand{\des}{\project{DES}}
\newcommand{\euclid}{\observatory{Euclid}}
\newcommand{\EA}{\observatory{Exoplanet Archive}}
\newcommand{\gaia}{\observatory{Gaia}}
\newcommand{\Gaia}{\gaia}
\newcommand{\galex}{\observatory{GALEX}}
\newcommand{\Kepler}{\observatory{Kepler}}
\newcommand{\kepler}{\Kepler}
\newcommand{\lsst}{\observatory{LSST}}
\newcommand{\MAST}{\observatory{MAST}}
\newcommand{\sdss}{\observatory{SDSS}}
\newcommand{\Spitzer}{\observatory{Spitzer}}
\newcommand{\spitzer}{\Spitzer}
\newcommand{\TESS}{\observatory{TESS}}
\newcommand{\thetractor}{\package{The~Tractor}}
\newcommand{\wfirst}{\observatory{WFIRST}}
\newcommand{\latin}[1]{\textit{#1}}
\newcommand{\eg}{\latin{e.g.}}
\newcommand{\etal}{\latin{et~al.}}
\newcommand{\etc}{\latin{etc.}}
\newcommand{\ie}{\latin{i.e.}}
\newcommand{\vs}{\latin{vs.}}

%----- math shih
\newcommand{\given}{\,|\,}

%----- typeset journals
% \newcommand{\aj}{Astron.\,J.}
% \newcommand{\apj}{Astrophys.\,J.}
% \newcommand{\apjl}{Astrophys.\,J.\,Lett.}
% \newcommand{\apjs}{Astrophys.\,J.\,Supp.\,Ser.}
% \newcommand{\mnras}{Mon.\,Not.\,Roy.\,Ast.\,Soc.}
% \newcommand{\aap}{Astron.\,\&~Astrophys.}

%----- Tighten up paragraphs and lists
\setlength{\parskip}{0.0ex}
\setlength{\parindent}{0.2in}
\renewenvironment{itemize}{\begin{list}{$\bullet$}{%
  \setlength{\topsep}{0.0ex}%
  \setlength{\parsep}{0.0ex}%
  \setlength{\partopsep}{0.0ex}%
  \setlength{\itemsep}{0.0ex}%
  \setlength{\leftmargin}{1.0\parindent}}}{\end{list}}
\newcounter{actr}
\renewenvironment{enumerate}{\begin{list}{\scriptsize{\textbf{\textsf{(\arabic{actr})}}}}{%
  \usecounter{actr}%
  \setlength{\topsep}{0.0ex}%
  \setlength{\parsep}{0.0ex}%
  \setlength{\partopsep}{0.0ex}%
  \setlength{\itemsep}{0.0ex}%
  \setlength{\leftmargin}{1.0\parindent}}}{\end{list}}

%----- mess with paragraph spacing!
\makeatletter
\renewcommand\paragraph{\@startsection{paragraph}{4}{\z@}%
                                    {1ex}%
                                    {-1em}%
                                    {\normalfont\normalsize\bfseries}}
\makeatother

%----- Special Hogg list for references
  \newcommand{\hogglist}{%
    \rightmargin=0in
    \leftmargin=0.25in
    \topsep=0ex
    \partopsep=0pt
    \itemsep=0ex
    \parsep=0pt
    \itemindent=-1.0\leftmargin
    \listparindent=\leftmargin
    \settowidth{\labelsep}{~}
    \usecounter{enumi}
  }

%----- side-to-side figure macro
%------- make numbers add up to 94%
 \newlength{\figurewidth}
 \newlength{\captionwidth}
 \newcommand{\ssfigure}[3]{%
   \setlength{\figurewidth}{#2\textwidth}
   \setlength{\captionwidth}{\textwidth}
   \addtolength{\captionwidth}{-\figurewidth}
   \addtolength{\captionwidth}{-0.02\figurewidth}
   \begin{figure}[htb]%
   \begin{tabular}{cc}%
     \begin{minipage}[c]{\figurewidth}%
       \resizebox{\figurewidth}{!}{\includegraphics{#1}}%
     \end{minipage} &%
     \begin{minipage}[c]{\captionwidth}%
       \textsf{\caption[]{\footnotesize {#3}}}%
     \end{minipage}%
   \end{tabular}%
   \end{figure}}

%----- top-bottom figure macro
 \newlength{\figureheight}
 \setlength{\figureheight}{0.75\textheight}
 \newcommand{\tbfigure}[2]{%
   \begin{figure}[htp]%
   \resizebox{\textwidth}{!}{\includegraphics{#1}}%
   \textsf{\caption[]{\footnotesize {#2}}}%
   \end{figure}}

%----- deal with pdf page-size stupidity
\special{papersize=8.5in,11in}
\setlength{\pdfpageheight}{\paperheight}
\setlength{\pdfpagewidth}{\paperwidth}

% no more bad lines!
\sloppy\sloppypar

\pagestyle{empty}
\begin{document}

\section*{Facilities and Resources}

The NYU Department of Physics is part of the NYU Faculty of Arts \&
Science and includes the Center for Cosmology and Particle Physics and
the Center for Soft Matter Research.  A wide variety of sponsored
research activities take place in the Department.  These range from
the theoretical to the pragmatic, and include a broad spectrum of
interactions with such disciplines as biology, medicine, chemistry,
applied mathematics, and computer science.

In addition to benefitting from this research activity directly and
indirectly, as a member of the Department, the PI receives through the
Department staff support for clerical work, post-award grant support,
and for computing (expanded upon below).

\paragraph{Computing Equipment:}

The astrophysics group at NYU maintains a high-performance scientific
computing cluster and a rack of heterogeneous compute servers.  The
latter contain more than 100~Tb of disk space, most of which is filled
with astrophysical imaging data.  The Physics Department maintains a
state-of-the-art controlled-environment computer room that houses and
protects the computers that will be used in this project.

\paragraph{General computer Resources:}

The University has hired a Director of Scientific Computing for the
NYU Center for Cosmology and Particle Physics (the PI's home).  His
responsibilities include management of the cluster and the data
servers, and overall management and supervision of the Center's
computer system.  He will oversee the computer hardware used in this
project.

A variety of computing resources are available within the Physics
Department, including UNIX workstations, laptops, laser color
printers, etc.  The Department runs a network of several hundred
desktop workstations, file servers, in addition to the multi-processor
computational server mentioned above.

Computational needs are also supported through the University's
Academic Computing Services, a unit of the NYU-wide Information
Technology Services offering an additional wide range of computational
resources in support of research and instruction.  These include a
variety of computing platforms, including several high-performance
multi-CPU systems, and scientific software.  Consultants are available
to assist in the use of these resources.

\paragraph{Office Space:}

The offices of the personnel will all be located in the NYU Physics
Department.

\paragraph{Library Resources:}

In addition to an enormous book collection, the NYU libraries hold
current subscriptions to hundreds of hardcopy and electronic journals.
It provides access to all journals relevant to this project and such
databases as MathSciNet, the Web of Science (science citation index),
and the ACM Digital Library.

\paragraph{Experimental and Hardware Facilities:}

The Department has large experimental facilities, including machining,
imaging, and clean facilities, but none of these are directly relevant
to this project, except insofar as they are part of the rich
intellectual and research atmosphere.

\end{document}
