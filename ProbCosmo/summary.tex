\documentclass[12pt]{article}
\setlength{\headheight}{0ex} % must be *before* \input{hogg_nsf}
\setlength{\headsep}{0ex} % must be *before* \input{hogg_nsf}
\input{hogg_nsf}
\begin{document}\sloppy\sloppypar\thispagestyle{empty}

\noindent
\textbf{New Probabilistic Approaches to Cosmology}
\smallskip

New imaging and spectroscopic astronomical surveys present
increasingly ambitious challenges and opportunities for cosmology.
We want to measure smaller signals with larger numbers of galaxies
observed at lower signal-to-noise ratios.
These goals, in the long, run, will require us to perform data
analyses that are as information-preserving as possible.
This proposal is to create new methods for cosmological data analysis
that will bring the cosmology community to performing cosmological
inferences using not lossy, derived data products (N-sigma galaxy
catalogs, best-fit redshifts, correlation function point estimates),
but something much closer to the original imaging and spectroscopic
data.

The new methods will have to be informed by principles of
probabilistic inference, but also good applied-mathematics technology,
to make them practical.
The PI's group already has some successes in this area but there are
many more opportunities, in weak lensing and large-scale structure.

...Weak lensing p(galaxy)

...LSS p(z)

...LSS and WL GP density reconstruction and marginalization

\noindent
\textbf{Intellectual Merit:}

...Stage-ZZZ cosmological projects.

...Making maximal use of the outputs of new surveys, esp LSST.

...Making the impossible possible.

...Applied math and astrophysics, together at last

\noindent
\textbf{Broader Impacts:}

...General methods and open-source code that will benefit all cosmological projects

...Bringing inference concepts into cosmology.

...Pedagogical papers on MCMC and GPs for the beginning grad school or late undergrad level

\end{document}
