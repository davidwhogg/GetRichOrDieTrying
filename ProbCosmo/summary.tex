\documentclass[12pt]{article}
\setlength{\headheight}{0ex} % must be *before* \input{hogg_nsf}
\setlength{\headsep}{0ex} % must be *before* \input{hogg_nsf}
\input{hogg_nsf}
\begin{document}\sloppy\sloppypar\thispagestyle{empty}

\noindent
\textbf{New Probabilistic Methods for Observational Cosmology} % must be synchronized with description.tex
\smallskip

New imaging and spectroscopic astronomical surveys present
increasingly ambitious challenges and opportunities for cosmology.
We want to measure smaller signals with larger numbers of galaxies
observed at lower signal-to-noise ratios.
These goals, in the long, run, will require us to perform data
analyses that are as information-preserving as possible.
This proposal is to create new methods for cosmological data analysis
that will bring the cosmology community to performing cosmological
inferences using not lossy, derived data products (N-sigma galaxy
catalogs, best-fit redshifts, correlation function point estimates),
but something much closer to the original imaging and spectroscopic
data.
The new methods will have to be informed by principles of
probabilistic inference, but also good applied-mathematics technology,
to make them practical.

In weak lensing studies, 

...Weak lensing p(galaxy) and p(PSF)

...LSS p(z)

...LSS and WL GP density reconstruction and marginalization; deconvolution, etc.

\noindent
\textbf{Intellectual Merit:}
Stage-III cosmological projects (such as \sdss, \des)
and Stage-IV survey teams (such as \lsst, \euclid, and \wfirst)
are starting to see the limits of point estimates and rigid catalogs;
they are all looking at producing probabilistic information about redshifts,
galaxy shapes, and deblending and photometry.
The methods developed in this project will be among the first practical methods for
cosmological inference and large-scale structure measurement that can
make full and proper (justified) use of these probabilistic outputs.
For the first time, it will be possible to perform
simultaneous inference or refinement of catalog-level properties along with
large-scale structure and cosmological inferences.
Simultaneous inference is expected to significantly \emph{reduce statistical
biases in cosmological measurements},
and also reduce variance in catalog-level quantities.

In general, full probabilistic inference at the survey level is intractable computationally.
The methods developed here will make use of the PI's expertise in, and collaborations with,
applied mathematics to make the impossible possible.

\noindent
\textbf{Broader Impacts:}

...General methods and open-source code that will benefit all cosmological projects.
Feedback on probabilistic outputs; new standards for those...

...Pedagogical papers on MCMC and GPs for the beginning grad school or late undergrad level

...Hack days and hack weeks

...Bringing inference concepts into cosmology.  Bringing applied math.

\end{document}
